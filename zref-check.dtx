% \iffalse meta-comment
%
% File: zref-check.dtx
%
% This file is part of the LaTeX package "zref-check".
%
% Copyright (C) 2021  Gustavo Barros
%
% It may be distributed and/or modified under the conditions of the
% LaTeX Project Public License (LPPL), either version 1.3c of this
% license or (at your option) any later version.  The latest version
% of this license is in the file:
%
%    https://www.latex-project.org/lppl.txt
%
% and version 1.3 or later is part of all distributions of LaTeX
% version 2005/12/01 or later.
%
%
% This work is "maintained" (as per LPPL maintenance status) by
% Gustavo Barros.
%
% This work consists of the files zref-check.dtx,
%                                 zref-check.ins,
%                                 zref-check.tex,
%                                 zref-check-code.tex,
%           and the derived files zref-check.sty and
%                                 zref-check.pdf,
%                                 zref-check-code.pdf.
%
% The released version of this package is available from CTAN.
%
% -----------------------------------------------------------------------
%
% The development version of the package can be found at
%
%    https://github.com/gusbrs/zref-check
%
% for those people who are interested.
%
% -----------------------------------------------------------------------
%
% \fi
%
% \iffalse
%<*driver>
\documentclass{l3doc}
% Have \GetFileInfo pick up date and version data
\usepackage{zref-check}
\NewDocumentCommand\opt{m}{\texttt{#1}}
\MakeShortVerb{\|}
\begin{document}
  \DocInput{zref-check.dtx}
\end{document}
%</driver>
% \fi
%
%
% \begin{documentation}
%
% \section{Introduction}
%
% The \pkg{zref-check} package provides an user interface for making \LaTeX{}
% cross-references exploiting document contextual information to enrich the
% way the reference can be rendered, but at the same time ensuring the means
% that these cross-references can be done consistently with the document
% structure.
%
% The usual \LaTeX{} cross-reference is done by referring to a \texttt{label},
% associated with one or another document structural element, and this
% reference will typeset for you some content based on the information which
% is stored in that label.  \cs{zcheck}, the main user command of
% \pkg{zref-check}, has a somewhat different concept.  Instead of trying to
% provide the text to be typeset based on the contextual information,
% \cs{zcheck} lets the user supply an arbitrary text and specify one or more
% checks to be done on the label(s) being referred to.  If any of the checks
% fails, a warning is issued upon compilation, so that the user can go back to
% that cross-reference and correct it as needed, without having to rely on
% burdensome and error prone manual proof-reading.
%
% This grants a much increased flexibility for the cross-reference text, which
% means in practice that the writing style, the variety of expressions you may
% use for similar situations, does not need to be sacrificed for the
% convenience.  \cs{zcheck}'s cross-references do not need to ``feel''
% automated to be consistently checked.  Localization is also not an issue,
% since the cross-reference text is provided directly by the user.  Separating
% ``typesetting'' from ``checking'' also means there is a lot of document
% context we can leverage for this purpose (see Section~\ref{sec:checks}).
%
% A standard \LaTeX{} cross-reference is made to refer to specific numbered
% document elements -- chapters, sections, figures, tables, equations, etc.
% The cross-reference will normally produce that number (which is the
% element's ``id'') and, eventually, its ``type'' (the counter).  We may also
% refer to the page that element occurs and even its ``title'' (in which case,
% atypically, we may even get to refer to an unnumbered section, provided we
% also implicitly supply by some means the ``id'').
%
% For references to these usual specific document elements, \pkg{zref-check}
% caters for a particular kind of cross-reference which is common:
% \emph{relational} statements based on them.  \cs{zcheck} can typeset and
% meaningfully check cross-references such as ``above'', ``on the next page'',
% ``on the facing page'', ``on the previous section'', ``later on this
% chapter'' and so on.  After all, if your reference is being made on page 2
% and refers to something on the same page, ``on this page'' reads much better
% than ``on page~2''.  If you are writing chapter~4, ``on the previous
% chapter'' sounds nicer than ``on chapter~3''.
%
% However, there is yet another kind of ``looser'' cross-reference we
% routinely do in our documents.  Expressions such as ``previously'', ``as
% mentioned before'', ``as will be discussed'', and so on, are a powerful
% discursive instrument, which enriches the text, by offering hints to the
% arguments' threads, without necessarily pressing them too hard onto the
% reader.  So, we might not want to say ``on footnote 57, pag.~34'', but
% prefer ``previously'', not ``on Section 3.4'', but rather ``below'', or
% ``later on''.  Besides, we also may refer to certain passages in the text in
% this way, rather than to numbered document elements.  And this kind of
% reference is not only hard to check and find, but also to fix.  After all,
% if you are making one such reference, you are taking that statement as a
% premisse at the current point in the text.  So, if that reference is
% missing, or relocated, you may need to bring in the support to the premisse
% for your argument to close, rather than just ``adjust the reference text''.
% \pkg{zref-check} also provides support for this kind of cross-reference,
% allowing for them to be consistently verified.
%
%
%
% \section{Loading the package}
% \label{sec:loading-package}
%
% As usual:
%
% \begin{syntax}
%   \cs{usepackage}\oarg{options}|{zref-check}|
% \end{syntax}
%
%
%
% \section{Dependencies}
%
% \pkg{zref} is required, of course, but in particular, its modules
% \pkg{zref-user} and \pkg{zref-abspage} are loaded by default.  \pkg{ifdraft}
% (from the \pkg{oberdiek} bundle) is also loaded by default.  A \LaTeX{}
% kernel later than 2021-06-01 is required, since we rely on the new hook
% system from \pkg{ltcmdhooks} for the sectioning checks.  If \pkg{hyperref}
% is loaded and option \pkg{hyperref} is given, \pkg{zref-check} makes use of
% it, but it does not load the package for you.
%
%
% \section{User interface}
% \label{sec:user-interface}
%
% \begin{function}{\zcheck}
%   \begin{syntax}
%     \cs{zcheck}\oarg{checks/options}\marg{labels}\marg{text}
%   \end{syntax}
%   Typesets \meta{text}, as given, while performing a list of \meta{checks}
%   on each of the \meta{labels}.  When \pkg{hyperref} support is enabled,
%   \meta{text} will be made a hyperlink to \emph{the first} \meta{label} in
%   \meta{labels}.  The starred version of the command does the same as the
%   plain one, just does not form a link.  The \meta{options} are (mostly) the
%   same as those of the package, and can be given to local effect.
%   \meta{checks} and \meta{options} can be given side by side as a comma
%   separated list in the optional argument.  \meta{labels} is also a comma
%   separated list.
% \end{function}
%
% \begin{function}{\zctarget}
%   \begin{syntax}
%     \cs{zctarget}\marg{label}\marg{text}
%   \end{syntax}
%   Typesets \meta{text}, as given, and places a pair of \texttt{zlabel}s, one
%   at the start of \meta{text}, using \meta{label} as label name, another one
%   (internal) at the end of \meta{text}.
% \end{function}
%
%
% \begin{function}{zcregion}
%   \begin{syntax}
%     |\begin{zcregion}|\marg{label}
%       |  ...|
%     |\end{zcregion}|
%   \end{syntax}
%   An environment that does the same as \cs{zctarget}, for cases of longer
%   stretches of text.
% \end{function}
%
%
% \begin{function}{\zrefchecksetup}
%   \begin{syntax}
%     \cs{zrefchecksetup}\marg{options}
%   \end{syntax}
%   Sets \pkg{zref-check}'s options (see Section~\ref{sec:options}).
% \end{function}
%
%
% \section{Checks}
% \label{sec:checks}
%
% \pkg{zref-check} provides several ``checks'' to be used with \cs{zcheck}.
% The checks may be combined in a \cs{zcheck} call, e.g.\ \opt{[close,
% after]}, or \opt{[thischap, before]}.  In this case, each check in
% \meta{checks} is performed against each of the \meta{labels}.  This is done
% independently for each check, which means, in practice, that the checks bear
% a logical \texttt{AND} relation to the others.  Whether the combination is
% meaningful, is up to the user.  As is the correspondence between the
% \meta{checks} and the \meta{text} in \cs{zcheck}.
%
% The use of checks which perform ``within the page'' comparisons -- namely
% \opt{above} and \opt{below} and, through them, \opt{before} and \opt{after}
% -- comes with some caveats you should be acquainted with.
% Section~\ref{sec:within-page-checks} discusses their limitations and expands
% on the expected workflow for their use to ensure reliable results.
%
% Note that the naming convention of the checks adopts the perspective of
% \cs{zcheck}.  That is, the name of the check describes the position of the
% label being referred to, relative to the \cs{zcheck} call being made.  For
% example, the \opt{before} check should issue no message if
% \cs{ztarget}|{mylabel}{...}| occurs before
% \cs{zcheck}|[before]{mylabel}{...}|.
%
% The available checks are the following:
%
% \begin{description}[leftmargin=0pt, itemindent=0pt, align=right,
%   labelsep=1em, font=\MacroFont]
%
% \item[thispage] \meta{label} occurs on the same page as \cs{zcheck}.
%
% \item[prevpage] \meta{label} occurs on the previous page relative to
%   \cs{zcheck}.
%
% \item[nextpage] \meta{label} occurs on the next page relative to
%   \cs{zcheck}.
%
% \item[facing] On a \texttt{twoside} document, both \meta{label} and
%   \cs{zcheck} fall onto a double spread, each on one of the two facing
%   pages.
%
% \item[above] \meta{label} and \cs{zcheck} are both on the same page, and
%   \meta{label} occurs ``above'' \cs{zcheck}.
%
% \item[below] \meta{label} and \cs{zcheck} are both on the same page, and
%   \meta{label} occurs ``below'' \cs{zcheck}.
%
% \item[pagesbefore] \meta{label} occurs on any page before the one of
%   \cs{zcheck}.
%
% \item[ppbefore] Convenience alias for \opt{pagesbefore}.
%
% \item[pagesafter] \meta{label} occurs on any page after the one of
%   \cs{zcheck}.
%
% \item[ppafter] Convenience alias for \opt{pagesafter}.
%
% \item[before] Either \opt{above} or \opt{pagesbefore}.
%
% \item[after]  Either \opt{below} or \opt{pagesafter}.
%
% \item[thischap] \meta{label} occurs on the same chapter as \cs{zcheck}.
%
% \item[prevchap] \meta{label} occurs on the previous chapter relative to the
%   one of \cs{zcheck}.
%
% \item[nextchap] \meta{label} occurs on the next chapter relative to the one
%   of \cs{zcheck}.
%
% \item[chapsbefore] \meta{label} occurs on any chapter before the one of
%   \cs{zcheck}.
%
% \item[chapsafter] \meta{label} occurs on any chapter after the one of
%   \cs{zcheck}.
%
% \item[thissec] \meta{label} occurs on the same section as \cs{zcheck}.
%
% \item[prevsec] \meta{label} occurs on the previous section (of the same
%   chapter) relative to the one of \cs{zcheck}.
%
% \item[nextsec] \meta{label} occurs on the next section (of the same chapter)
%   relative to the one of \cs{zcheck}.
%
% \item[secsbefore] \meta{label} occurs on any section (of the same chapter)
%   before the one of \cs{zcheck}.
%
% \item[secsafter] \meta{label} occurs on any section (of the same chapter)
%   after the one of \cs{zcheck}.
%
% \item[close] \meta{label} occurs within a page range from \opt{closerange}
%   pages before the one of \cs{zcheck} to \opt{closerange} pages after it
%   (about the \opt{closerange} option, see Section~\ref{sec:options}).
%
% \item[far] Not \opt{close}.
%
% \end{description}
%
%
%
% \section{Options}
% \label{sec:options}
%
% Options are a standard \texttt{key=value} comma separated list, and can be
% set globally either as \cs{usepackage}\oarg{options} at load-time (see
% Section~\ref{sec:loading-package}), or by means of \cs{zrefchecksetup} (see
% Section~\ref{sec:user-interface}) in the preamble.  Most options can also be
% used with local effects, through the optional argument of \cs{zcheck}.
%
% \DescribeOption{hyperref}
% Controls the use of \pkg{hyperref} by \pkg{zref-check} and takes values
% \opt{auto}, \opt{true}, \opt{false}.  The default value, \opt{auto}, makes
% \pkg{zref-check} use \pkg{hyperref} if it is loaded, meaning \cs{zcheck} can
% be hyperlinked to the \emph{first label} in \meta{labels}.  \opt{true} does
% the same thing, but warns if \pkg{hyperref} is not loaded (\pkg{hyperref} is
% never loaded for you).  In either of these cases, if \pkg{hyperref} is
% loaded, module \pkg{zref-hyperref} is also loaded by \pkg{zref-check}.
% \opt{false} means not to use \pkg{hyperref} regardless of its availability.
% This is a preamble only option, but \cs{zcheck} provides granular control of
% hyperlinking by means of its starred version.
%
%
% \DescribeOption{msglevel}
% Sets the level of messages issued by \cs{zcheck} failed checks and takes
% values \opt{warn}, \opt{info}, \opt{none}, \opt{obeydraft}, \opt{obeyfinal}.
% The default value, \opt{warn}, issues messages both to the terminal and to
% the log file, \opt{info} issues messages to the log file only, \opt{none}
% suppresses all messages.  \opt{obeydraft} corresponds to \opt{info} if
% option \opt{draft} is passed to \cs{documentclass}, and to \opt{warn}
% otherwise.  \opt{obeyfinal} corresponds to \opt{warn} if option \opt{final}
% is (explicitly) passed to \cs{documentclass} and \opt{info} otherwise.
% \opt{ignore} is provided as convenience alias for \opt{msglevel=none} for
% local use only.  This option only affects the messages issued by the checks
% in \cs{zcheck}, not other messages or warnings of the package.  In
% particular, it does not affect warnings issued for undefined labels, which
% just use \cs{zref@refused} and thus are the same as standard \LaTeX{} ones
% for this purpose.
%
%
% \DescribeOption{onpage}
% Allows to control the messaging style for ``within page checks'', and takes
% values \opt{labelseq}, \opt{msg}, \opt{obeydraft}, \opt{obeyfinal}.  The
% default, \opt{labelseq}, uses the labels' shipout sequence, as retrieved
% from the \file{.aux} file, to infer relative position within the page.
% \opt{msg} also uses the same method for checking relative position, but
% issues a (different) message \emph{even if the check passes}, to provide a
% simple workflow for robust checking of ``false negatives'', considering the
% label sequence is not fool proof (for details and workflow recommendations,
% see Section~\ref{sec:within-page-checks}).  \opt{msg} also issues its
% messages at the same level defined in \opt{msglevel}.  \opt{obeydraft}
% corresponds to \opt{labelseq} if option \opt{draft} is passed to
% \cs{documentclass} and to \opt{msg} otherwise.  \opt{obeyfinal} corresponds
% to \opt{msg} if option \opt{final} is (explicitly) passed to
% \cs{documentclass}, and to \opt{labelseq} otherwise.
%
%
% \DescribeOption{closerange}
% Defines the width of the range of pages, relative to the reference, that are
% considered ``close'' by the \opt{close} check.  Takes a positive integer as
% value, with default 5.
%
%
% \DescribeOption{labelcmd}
% Defines the command used to set the user labels in \cs{zctarget} and
% \texttt{zcregion}.  Takes a control sequence \emph{name} as value, and the
% default sets labels with the minimal required properties, those of the
% \texttt{zrefcheck} property list.  This is a preamble only option.  The
% specified control sequence must receive one mandatory argument (the
% \marg{label}) and must generate a \texttt{zref label} with at least the
% properties in the \texttt{zrefcheck} property list.  The intended use case
% is that of the user creating a convenience macro which calls both \cs{label}
% and \cs{zlabel}, as suggested in Section~\ref{sec:labels}, so that the same
% labels are accessible either from the standard reference system or from
% \pkg{zref}.  For example:
%
% \begin{verbatim}
%     \NewDocumentCommand\mybothlabels{m}{\label{#1}\zlabel{#1}}
%     \zrefchecksetup{labelcmd=mybothlabels}
% \end{verbatim}
%
% Note that the value of the underlying counter used for labels in
% \cs{zctarget} and \texttt{zcregion} -- what you'd get with a plain \cs{ref}
% here -- is not really meaningful.  But you get to use
% \cs{pageref}\marg{label}, or \pkg{hyperref}'s
% \cs{hyperref}\oarg{label}\marg{text} on the labels used in \cs{zctarget} and
% \texttt{zcregion} with this procedure.
%
%
% \section{Labels}
% \label{sec:labels}
%
% \pkg{zref-check} depends on \pkg{zref}, as the name entails, which means it
% is able to work with \pkg{zref} labels, in general created by \cs{zlabel},
% but also with \cs{zctarget} and the \texttt{zcregion} environment provided
% by this package.  This has some advantages, particularly the data
% flexibility of \pkg{zref}, and the absence of the ubiquitous ``load-order''
% and compatibility problems which are well known to afflict \LaTeX{} packages
% of this area of functionality.  On the other hand, the reliance on
% \pkg{zref} labels may be seen as an inconvenience, since users of the
% standard cross-reference infrastructure may need to add extra labels for
% this.  That's true.  But \pkg{zref-check} is not meant to replace the
% existing functionality of the kernel or of the traditional packages in this
% area (to my knowledge, it only intersects directly with \pkg{varioref} and,
% even so, it is quite different in scope).  Indeed, it is easy to see the use
% in tandem with standard references, for example:
%
% \begin{verbatim}
%     ... Figure~\ref{fig:figure-1}, \zcheck*[nextpage]{fig:figure-1}{on
%     the next page}.
% \end{verbatim}
%
% Besides, \pkg{zref} does not share the label name-space with the standard
% labels, so that you can call both \cs{label} and \cs{zlabel} with the same
% label name (manually, or through a convenience macro), to ease the label set
% administration.  The example above presumes that was the case.
%
% All user commands of \pkg{zref-check} have their \marg{label} arguments
% protected for \pkg{babel} active characters using \pkg{zref}'s
% \cs{zref@wrapper@babel}, so that we should have equivalent support in that
% regard, as \pkg{zref} itself does.  However, \pkg{zref-check} sets labels
% which either start with \texttt{zrefcheck@} or end with \texttt{@zrefcheck},
% for internal use.  Label names with either of those are considered reserved
% by the package.
%
%
% \section{Limitations}
%
% There are three qualitatively different kinds of checks being used by
% \cs{zcheck}, according to the source and reliability of the information they
% mobilize: page number checks, within page checks, and sectioning checks.
%
%
% \subsection{Page number checks}
%
% Page number checks -- \opt{thispage}, \opt{prevpage}, \opt{nextpage},
% \opt{pagesbefore}, \opt{pagesafter}, \opt{facing} -- use the
% \texttt{abspage} property provided by the \pkg{zref-abspage} module.  This
% is a solid piece of information, on which we can rely upon.  However,
% despite that, page number checks may still become ill-defined, if the
% \meta{text} argument in \cs{zcheck}, when typeset, crosses page boundaries,
% starting in one page, and finishing in another.  The same can happen with
% the text in \cs{zctarget} and the \texttt{zcregion} environment.
%
% This is why the user commands of this package set a pair or labels around
% \meta{text}.  So, when checking \cs{zcheck} against a regular
% \texttt{zlabel} both the start and the end of the \meta{text} are checked
% against the label, and the check fails if either of them fails.  When
% checking \cs{zcheck} against a \cs{zctarget} or a \texttt{zcregion}, both
% beginnings and ends are checked against each other two by two, and if any of
% them fails, the check fails.  In other words, if a page number checks
% passes, we know that the entire \meta{text} arguments pass it.
%
% This is a corner case (albeit relevant) which must be taken care of, and it
% is possible to do so robustly.  Hence, we can expect reliable results in
% these tests.
%
%
% \subsection{Within page checks}
% \label{sec:within-page-checks}
%
% When both label and reference fall on the same page things become much
% trickier.  This is basically the case of the checks \opt{above} and
% \opt{below} (and, through them, \opt{before} and \opt{after}).  There is no
% equally reliable information (that I know of) as we have for the page number
% checks for this, especially when floats come into play.  Which, of course,
% is the interesting case to handle.
%
% To infer relative position of label and reference on the same page,
% \pkg{zref-check} uses the labels' shipout sequence, which is retrieved at
% load-time from the order in which the labels occur in the \file{.aux} file.
% Indeed, \pkg{zref} writes labels to the \file{.aux} file at shipout (and,
% hence, in shipout order), and needs to do so, because a number of its
% properties are only available at that point.
%
% However, even if this method will buy us a correct check for a regular float
% on a regular page (which, to be fair, is a good result), it is not difficult
% do conceive situations in which this sequence may not be meaningful, or even
% correct, for the case.  A number of cases which may do so are: two column
% documents, text wrapping, scaling, overlays, etc.  (I don't know if those
% make the method fail, I just don't know if they don't).  Therefore, the
% \texttt{labelseq} should be taken as a \emph{proxy} and not fully reliable,
% meaning that the user should be watchful of its results.
%
% For this reason, \pkg{zref-check} provides an easy way to do so, by allowing
% specific control of the messaging style of the checks which do within page
% comparisons though the option \opt{onpage}.  The concern is not really with
% false positives (getting a warning when it was not due), but with false
% negatives (not getting a warning when it was due).  Hence, setting
% \opt{onpage} to \opt{msg} at a final typesetting stage (or just set it to
% \opt{obeydraft} or \opt{obeyfinal} if that's part of your workflow) provides
% a way to easily identify all cases of such checks (failing or passing), and
% double-check them.  In case the test is passing though, the message is
% different from that of a failing check, to quickly convey why you are
% getting the message.  This option can also be set at the local level, if the
% page in question is known to be problematic, or just atypical.
%
%
% \subsection{Sectioning checks}
%
% The information used by sectioning checks is provided by means of dedicated
% counters for chapters and sections, similarly as standard counters for them,
% but which are stepped and reset regardless of whether these sectioning
% commands are numbered or not (that is, starred or not).  And this for two
% reasons.  First, we don't need the absolute counter value to be able to make
% the kind of relative statement we want to do here.  Second, this allows us
% to have these checks work for numbered and unnumbered sectioning commands
% without having to worry about how those are used within the document.
%
% The caveat is that the package does this by hooking into \cs{chapter} and
% \cs{section}, which poses two restrictions for the proper working of these
% checks.  First, we are using the new hook system for this, as provided by
% \pkg{ltcmdhooks}, which means a \LaTeX{} kernel later than 2021-06-01 is
% required.  Second, since we are hooking into \cs{chapter} and \cs{section},
% these checks presume these commands are being used by the document class for
% this purpose (either directly, or internally as, for example, KOMA-Script's
% \cs{addchap} and \cs{addsec} do).  If that's not the case, additional setup
% may be needed for these checks to work as expected.
%
%
%
% \section{Change history}
%
% A change log with relevant changes for each version, eventual upgrade
% instructions, and upcoming changes, is maintained in the package's
% repository, at
% \url{https://github.com/gusbrs/zref-check/blob/main/CHANGELOG.md}.
%
%
%
% \end{documentation}
%
%
% \begin{implementation}
%
% \section{Initial setup}
%
% Start the \pkg{DocStrip} guards.
%    \begin{macrocode}
%<*package>
%    \end{macrocode}
%
% Identify the internal prefix (\LaTeX3 \pkg{DocStrip} convention).
%    \begin{macrocode}
%<@@=zrefcheck>
%    \end{macrocode}
%
%
% For the \texttt{chapter} and \texttt{section} checks, \pkg{zref-check} uses
% the new hook system in \pkg{ltcmdhooks}, which was released with the
% 2021/06/01 \LaTeX{} kernel.
%    \begin{macrocode}
\providecommand\IfFormatAtLeastTF{\@ifl@t@r\fmtversion}
\IfFormatAtLeastTF{2021-06-01}
  {}
  {%
    \PackageError{zref-check}{LaTeX kernel too old}
      {%
        'zref-check' requires a LaTeX kernel newer than 2021-06-01.%
        \MessageBreak Loading will abort!%
      }%
    \endinput
  }%
%    \end{macrocode}
%
% Identify the package.
%    \begin{macrocode}
\ProvidesExplPackage {zref-check} {2021-08-17} {0.2.0}
  {Flexible cross-references with contextual checks based on zref}
%    \end{macrocode}
%
% \section{Dependencies}
%
%    \begin{macrocode}
\RequirePackage { zref-user }
\RequirePackage { zref-abspage }
\RequirePackage { ifdraft }
%    \end{macrocode}
%
%
% \section{\pkg{zref} setup}
%
% \begin{variable}{\g_@@_abschap_int, \g_@@_abssec_int}
%   Provide absolute counters for section and chapter, and respective
%   \pkg{zref} properties, so that we can make checks about relation of
%   chapters/sections regardless of internal counters, since we don't get
%   those for the unnumbered (starred) ones.  About the proper place to make
%   the hooks for this purpose, see
%   \url{https://tex.stackexchange.com/q/605533/105447} (thanks Ulrike
%   Fischer).
%    \begin{macrocode}
\int_new:N \g_@@_abschap_int
\int_new:N \g_@@_abssec_int
%    \end{macrocode}
% \end{variable}
%
%
% If the documentclass does not define \cs{chapter} the only thing that
% happens is that the chapter counter is never incremented, and the section
% one never reset.
%    \begin{macrocode}
\AddToHook { cmd / chapter / before }
  {
    \int_gincr:N \g_@@_abschap_int
    \int_zero:N \g_@@_abssec_int
  }
\zref@newprop { zc@abschap } [0] { \int_use:N \g_@@_abschap_int }
\zref@addprop \ZREF@mainlist { zc@abschap }
%    \end{macrocode}
%
%    \begin{macrocode}
\AddToHook { cmd / section / before }
  { \int_gincr:N \g_@@_abssec_int }
\zref@newprop { zc@abssec } [0] { \int_use:N \g_@@_abssec_int }
\zref@addprop \ZREF@mainlist { zc@abssec }
%    \end{macrocode}
%
%
% This is the list of properties to be used by \pkg{zref-check}, that is, the
% list of properties the references and targets store.  This is the minimum
% set required, more properties may be added according to options.
%    \begin{macrocode}
\zref@newlist { zrefcheck }
\zref@addprops { zrefcheck }
  {
    page ,
    abspage ,
    zc@abschap ,
    zc@abssec
  }
%    \end{macrocode}
%
%
% \section{Plumbing}
%
% \subsection{Messages}
%
% \begin{macro}{\@@_message:nnnn, \@@_message:nnnx}
%    \begin{macrocode}
\cs_new_protected:Npn \@@_message:nnnn #1#2#3#4
  {
    \use:c { msg_ \l_@@_msglevel_tl :nnnnn }
      { zref-check } {#1} {#2} {#3} {#4}
  }
\cs_generate_variant:Nn \@@_message:nnnn { nnnx }
%    \end{macrocode}
% \end{macro}
%
%    \begin{macrocode}
\msg_new:nnn { zref-check } { check-failed }
  { Failed~check~'#1'~for~label~'#2'~on~page~#3~\msg_line_context:. }
\msg_new:nnn { zref-check } { double-check }
  { Double-check~'#1'~for~label~'#2'~on~page~#3~\msg_line_context:. }
%    \end{macrocode}
%
%    \begin{macrocode}
\msg_new:nnn { zref-check } { check-missing }
  { Check~'#1'~not~defined~\msg_line_context:. }
\msg_new:nnn { zref-check } { property-undefined }
  { Property~'#1'~not~defined~\msg_line_context:. }
\msg_new:nnn { zref-check } { property-not-in-label }
  { Label~'#1'~has~no~property~'#2'~\msg_line_context:. }
\msg_new:nnn { zref-check } { property-not-integer }
  { Property~'#1'~for~label~'#2'~not~an~integer~\msg_line_context:. }
%    \end{macrocode}
%
%    \begin{macrocode}
\msg_new:nnn { zref-check } { hyperref-preamble-only }
  {
    Option~'hyperref'~only~available~in~the~preamble. \iow_newline:
    Use~the~starred~version~of~'\noexpand\zcheck'~instead.
  }
\msg_new:nnn { zref-check } { missing-hyperref }
  { Missing~'hyperref'~package. \iow_newline: Setting~'hyperref=false'. }
\msg_new:nnn { zref-check } { ignore-document-only }
  {
    Option~'ignore'~only~available~in~the~document. \iow_newline:
    Use~option~'msglevel'~instead.
  }
\msg_new:nnn { zref-check } { option-preamble-only }
  { Option~'#1'~only~available~in~the~preamble~\msg_line_context:. }
\msg_new:nnn { zref-check } { closerange-not-positive-integer }
  {
    Option~'closerange'~not~a~positive~integer~\msg_line_context:.~
    Using~default~value.
  }
\msg_new:nnn { zref-check } { labelcmd-undefined }
  {
    Control~sequence~named~'#1'~used~in~option~'labelcmd'~is~not~defined.~
    Using~default~value.
  }
%    \end{macrocode}
%
%
% \subsection{Integer testing}
%
% \begin{macro}{\@@_is_integer:n, \@@_int_to_roman:w}
%   From \url{https://tex.stackexchange.com/a/244405} (thanks Enrico Gregorio,
%   aka `egreg'), also see \url{https://tex.stackexchange.com/a/19769}.
%   Following the \texttt{l3styleguide}, I made a copy of
%   \cs{__int_to_roman:w}, since it is an internal function from the
%   \texttt{int} module, but we still get a warning from \texttt{l3build doc},
%   complaining about it.  And we're using \cs{tl_if_empty:oTF} instead of
%   \cs{tl_if_blank:oTF} as in egreg's answer, since \cs{romannumeral} is
%   defined so that ``the expansion is empty if the number is zero or
%   negative'', not ``blank''.  A couple of comments about this technique: the
%   underlying \cs{romannumeral} ignores space tokens and explicit signs
%   (\texttt{+} and \texttt{-}) in the expansion and hence it can only be used
%   to test positive integers; also the technique cannot distinguish whether
%   it received an empty argument or if ``the expansion was empty'' as a
%   result of receiving number as argument, so this must also be controlled
%   for since, in our use case, this may happen.
%    \begin{macrocode}
\cs_new_eq:NN \@@_int_to_roman:w \__int_to_roman:w
\prg_new_conditional:Npnn \@@_is_integer:n #1 { p, T , F , TF }
  {
    \tl_if_empty:oTF {#1}
      { \prg_return_false: }
      {
        \tl_if_empty:oTF { \@@_int_to_roman:w -0#1 }
          { \prg_return_true:  }
          { \prg_return_false: }
      }
  }
%    \end{macrocode}
% \end{macro}
%
% \begin{macro}{\@@_is_integer_rgx:n}
%   A possible alternative to \cs{@@_is_integer:n} is to use a straightforward
%   regexp match (see \url{https://tex.stackexchange.com/a/427559}).  It does
%   not suffer from the mentioned caveats from the \cs{__int_to_roman:w}
%   technique, however, while \cs{@@_is_integer:n} is expandable,
%   \cs{@@_is_integer_rgx:n} is not.  Also, \cs{@@_is_integer_rgx:n} is
%   probably slower.
%    \begin{macrocode}
\prg_new_protected_conditional:Npnn \@@_is_integer_rgx:n #1 { TF }
  {
    \regex_match:nnTF { \A\d+\Z } {#1}
      { \prg_return_true:  }
      { \prg_return_false: }
  }
%    \end{macrocode}
% \end{macro}
%
%
%
% \subsection{Options}
%
%
% \subsubsection*{\opt{hyperref} option}
%
% \begin{variable}{\l_@@_use_hyperref_bool, \l_@@_warn_hyperref_bool}
%    \begin{macrocode}
\bool_new:N \l_@@_use_hyperref_bool
\bool_new:N \l_@@_warn_hyperref_bool
\keys_define:nn { zref-check }
  {
    hyperref .choice: ,
    hyperref / auto .code:n =
      {
        \bool_set_true:N \l_@@_use_hyperref_bool
        \bool_set_false:N \l_@@_warn_hyperref_bool
      } ,
    hyperref / true .code:n =
      {
        \bool_set_true:N \l_@@_use_hyperref_bool
        \bool_set_true:N \l_@@_warn_hyperref_bool
      } ,
    hyperref / false .code:n =
      {
        \bool_set_false:N \l_@@_use_hyperref_bool
        \bool_set_false:N \l_@@_warn_hyperref_bool
      } ,
    hyperref .initial:n = auto ,
    hyperref .default:n = auto
  }
%    \end{macrocode}
% \end{variable}
%
%    \begin{macrocode}
\AddToHook { begindocument }
  {
    \@ifpackageloaded { hyperref }
      {
        \bool_if:NT \l_@@_use_hyperref_bool
          {
            \RequirePackage { zref-hyperref }
            \zref@addprop { zrefcheck } { anchor }
          }
      }
      {
        \bool_if:NT \l_@@_warn_hyperref_bool
          { \msg_warning:nn { zref-check } { missing-hyperref } }
        \bool_set_false:N \l_@@_use_hyperref_bool
      }
    \keys_define:nn { zref-check }
      {
        hyperref .code:n =
          { \msg_warning:nn { zref-check } { hyperref-preamble-only } }
      }
  }
%    \end{macrocode}
%
%
% \subsubsection*{\opt{msglevel} option}
%
% \begin{variable}{\l_@@_msglevel_tl}
%    \begin{macrocode}
\tl_new:N \l_@@_msglevel_tl
\keys_define:nn { zref-check }
  {
    msglevel .choice: ,
    msglevel / warn .code:n =
      { \tl_set:Nn \l_@@_msglevel_tl { warning } } ,
    msglevel / info .code:n =
      { \tl_set:Nn \l_@@_msglevel_tl { info } } ,
    msglevel / none .code:n =
      { \tl_set:Nn \l_@@_msglevel_tl { none } } ,
    msglevel / obeydraft .code:n =
      {
        \ifdraft
          { \tl_set:Nn \l_@@_msglevel_tl { info } }
          { \tl_set:Nn \l_@@_msglevel_tl { warning } }
      } ,
    msglevel / obeyfinal .code:n =
      {
        \ifoptionfinal
          { \tl_set:Nn \l_@@_msglevel_tl { warning } }
          { \tl_set:Nn \l_@@_msglevel_tl { info } }
      } ,
    msglevel .value_required:n = true ,
    msglevel .initial:n = warn ,
%    \end{macrocode}
% \opt{ignore} is a convenience alias for \opt{msglevel=none}, but only for
% use in the document body.
%    \begin{macrocode}
    ignore .code:n =
      { \msg_warning:nn { zref-check } { ignore-document-only } } ,
    ignore .value_forbidden:n = true
  }
%    \end{macrocode}
% \end{variable}
%
%    \begin{macrocode}
\AddToHook { begindocument }
  {
    \keys_define:nn { zref-check }
      { ignore .meta:n = { msglevel = none } }
  }
%    \end{macrocode}
%
%
% \subsubsection*{\opt{onpage} option}
%
% \begin{variable}{\l_@@_msgonpage_bool}
%    \begin{macrocode}
\bool_new:N \l_@@_msgonpage_bool
\keys_define:nn { zref-check }
  {
    onpage .choice: ,
    onpage / labelseq .code:n =
      {
        \bool_set_false:N \l_@@_msgonpage_bool
      } ,
    onpage / msg .code:n =
      {
        \bool_set_true:N \l_@@_msgonpage_bool
      } ,
    onpage / obeydraft .code:n =
      {
        \ifdraft
          { \bool_set_false:N \l_@@_msgonpage_bool }
          { \bool_set_true:N \l_@@_msgonpage_bool }
      } ,
    onpage / obeyfinal .code:n =
      {
        \ifoptionfinal
          { \bool_set_true:N \l_@@_msgonpage_bool }
          { \bool_set_false:N \l_@@_msgonpage_bool }
      } ,
    onpage .value_required:n = true ,
    onpage .initial:n = labelseq
  }
%    \end{macrocode}
% \end{variable}
%
%
% \subsubsection*{\opt{closerange} option}
%
% \begin{variable}{\l_@@_close_range_int}
%    \begin{macrocode}
\int_new:N \l_@@_close_range_int
\keys_define:nn { zref-check }
  {
    closerange .code:n =
      {
        \@@_is_integer_rgx:nTF {#1}
          { \int_set:Nn \l_@@_close_range_int { \int_eval:n {#1} } }
          {
            \msg_warning:nn { zref-check } { closerange-not-positive-integer }
            \int_set:Nn \l_@@_close_range_int { 5 }
          }
      } ,
    closerange .value_required:n = true ,
    closerange .initial:n = 5
  }
%    \end{macrocode}
% \end{variable}
%
%
% \subsubsection*{\opt{labelcmd} option}
%
% \begin{variable}{\l_@@_target_label_tl}
%   I'd love to receive the macro itself rather than it's name, but this would
%   bring unwarranted complications:
%   \url{https://tex.stackexchange.com/a/489570}.
%    \begin{macrocode}
\tl_new:N \l_@@_target_label_tl
\bool_new:N \l_@@_target_label_bool
\keys_define:nn { zref-check }
  {
    labelcmd .code:n =
      {
        \tl_set:Nn \l_@@_target_label_tl {#1}
        \bool_set_true:N \l_@@_target_label_bool
      } ,
    labelcmd .value_required:n = true ,
  }
%    \end{macrocode}
% \end{variable}
%
% \begin{macro}{\@@_target_label:n}
%   Default definition of the function for user label setting in \cs{zctarget}
%   and \texttt{zcregion}.  It may be redefined at \texttt{begindocument}
%   according to option \opt{labelcmd}.
%    \begin{macrocode}
\cs_new_protected:Npn \@@_target_label:n #1
  { \zref@labelbylist {#1} { zrefcheck } }
%    \end{macrocode}
% \end{macro}
%
%    \begin{macrocode}
\AddToHook { begindocument }
  {
    \bool_if:NT \l_@@_target_label_bool
      {
        \tl_if_blank:VT \l_@@_target_label_tl
          { \tl_clear:N \l_@@_target_label_tl }
        \cs_if_exist:cTF { \l_@@_target_label_tl }
          {
            \cs_set_protected:Npx \@@_target_label:n #1
              {
                \exp_not:o
                  { \cs:w \l_@@_target_label_tl \cs_end: }
                  {#1}
              }
          }
          {
            \exp_args:Nnno \msg_warning:nnn { zref-check }
              { labelcmd-undefined } { \l_@@_target_label_tl }
          }
      }
    \keys_define:nn { zref-check }
      {
        labelcmd .code:n =
          {
            \msg_warning:nnn { zref-check }
              { option-preamble-only } { labelcmd }
          }
      }
  }
%    \end{macrocode}
%
%
% \subsubsection*{Package options}
%
% Process load-time package options
% (\url{https://tex.stackexchange.com/a/15840}).
%    \begin{macrocode}
\RequirePackage { l3keys2e }
\ProcessKeysOptions { zref-check }
%    \end{macrocode}
%
%
% \begin{macro}[int]{\zrefchecksetup}
%   Provide \cs{zrefchecksetup}.
%    \begin{macrocode}
\NewDocumentCommand \zrefchecksetup { m }
  { \keys_set:nn { zref-check } {#1} }
%    \end{macrocode}
% \end{macro}
%
%
%
% \subsection{Position on page}
%
% Method for determining relative position within the page: the sequence in
% which the labels get shipped out, inferred from the sequence in which the
% labels occur in the \file{.aux} file.
%
% Some relevant info about the sequence of things:
% \url{https://tex.stackexchange.com/a/120978} and \texttt{texdoc lthooks},
% section ``Hooks provided by |\begin{document}|''.
%
%
% One first attempt at this was to use \cs{zref@newlabel}, which is the macro
% in which \pkg{zref} stores the label information in the aux file.  When the
% \file{.aux} file is read at the beginning of the compilation, this macro is
% expanded for each of the labels.  So, by redefining this macro we can feed a
% variable (a L3 sequence), and then do what it usually does, which is to
% define each label with the internal macro \cs{@newl@bel}, when the
% \file{.aux} file is read.
%
% Patching this macro for this is not possible.  First, \cs{zref@newlabel} is
% one of those ``commands that look ahead'' mentioned in \pkg{ltcmdhooks}
% documentation.  Indeed, \cs{@newl@bel} receives 3 arguments, and
% \cs{zref@newlabel} just passes the first, the following two will be scanned
% ahead.  Second, the \pkg{ltcmdhooks} hooks are not actually available when
% the \file{.aux} file is read, they come only after |\begin{document}|.
% Hence, redefinition would be the only alternative.  My attempts at this
% ended up registered at \url{https://tex.stackexchange.com/a/604744}.  But
% the best result in these lines was:
%
% \begin{verbatim}
% \ZREF@Robust\edef\zref@newlabel#1{
%   \noexpand\seq_gput_right:Nn \noexpand\g__zrefcheck_auxfile_lblseq_seq {#1}
%   \noexpand\@newl@bel{\ZREF@RefPrefix}{#1}
% }
% \end{verbatim}
%
%
% However, better than the above is to just read it from the \file{.aux} file
% directly, which relieves us from hacking into any internals.  That's what
% David Carlisle's answer at \url{https://tex.stackexchange.com/a/147705}
% does.  This answer has actually been converted into the package
% \pkg{listlbls} by Norbert Melzer, but it is made to work with regular
% labels, not with \pkg{zref}'s.  And it also does not really expose the
% information in a retrievable way (as far as I can tell).  So, the below is
% adapted from Carlisle's answer's technique (a poor man's version of it...).
%
% There is some subtlety here as to whether this approach makes it safe for us
% to read the labels at this point without \cs{zref@wrapper@babel}.  The
% common wisdom is that babel's shorthands are only active after
% |\begin{document}| (e.g., \url{https://tex.stackexchange.com/a/98897}).
% Alas, it is more complicated than that.  Babel's documentation says (in
% section 9.5 Shorthands): ``To prevent problems with the loading of other
% packages after babel we reset the catcode of the character to the original
% one at the end of the package and of each language file (except with
% KeepShorthandsActive).  It is re-activate[d] again at |\begin{document}|.
% We also need to make sure that the shorthands are active during the
% processing of the \file{.aux} file.  Otherwise some citations may give
% unexpected results in the printout when a shorthand was used in the optional
% argument of |\bibitem| for example.''  This is done with
% |\if@filesw \immediate\write\@mainaux{...}|.  In other words, the
% catcode change is written in the \file{.aux} file itself!  Indeed, if you
% inspect the file, you'll find them there.  Besides, there is still the
% ominous ``except with KeepShorthandsActive''.
%
% However, the \emph{method} we're using here is not quite the same as the
% usual run of the \file{.aux} file, because we're actively discarding the
% lines for which the first token is not equal to \cs{zref@newlabel}.  I have
% tested the famous sensitive case for this: \pkg{babel} \opt{french} and
% labels with colons.  And things worked as expected.  Well, \emph{if}
% \opt{KeepShorthandsActive} is enabled \emph{with \opt{french}} and we load
% the package \emph{after babel} things do break, but not quite because of the
% colons in the labels.  Even \pkg{siunitx} breaks in the same
% conditions\dots{}
%
% For reference: About what are valid characters for use in labels:
% \url{https://tex.stackexchange.com/a/18312}.  About some problems with
% active colons: \url{https://tex.stackexchange.com/a/89470}.  About the
% difference between L3 strings and token lists, see
% \url{https://tex.stackexchange.com/a/446381}, in particular Joseph Wright's
% comment: ``Strings are for data that will never be typeset, for example file
% names, identifiers, etc.: if the material may be used in typesetting, it
% should be a token list.''  See also moewe's (CW) answer in the same lines.
% Which suggests using L3 strings for the reference labels might be a good
% catch all approach, and possibly more robust.  David Carlisle's comment
% about \pkg{inputenc} and how the strings work is a caveat (see
% \url{https://tex.stackexchange.com/q/446123#comment1516961_446381}, thanks
% David Carlisle).  Still\dots{} let's stick to tradition as long as it works,
% \pkg{zref} already does a great job in this regard anyway.
%
%
% \begin{variable}{\g_@@_auxfile_lblseq_prop}
%    \begin{macrocode}
\prop_new:N \g_@@_auxfile_lblseq_prop
%    \end{macrocode}
% \end{variable}
%
%    \begin{macrocode}
\tl_set:Nn \g_tmpa_tl { \c_sys_jobname_str .aux }
\file_if_exist:nT { \g_tmpa_tl }
  {
%    \end{macrocode}
% Retrieve the information from the \file{.aux} file, and store it in a
% property list, so that the sequence can be retrieved in key-value fashion.
%    \begin{macrocode}
    \ior_open:Nn \g_tmpa_ior { \g_tmpa_tl }
    \group_begin:
      \int_zero:N \l_tmpa_int
      \tl_clear:N \l_tmpa_tl
      \tl_clear:N \l_tmpb_tl
      \bool_set_false:N \l_tmpa_bool
      \ior_map_variable:NNn \g_tmpa_ior \l_tmpa_tl
        {
          \tl_map_variable:NNn \l_tmpa_tl \l_tmpb_tl
            {
              \tl_if_eq:NnTF \l_tmpb_tl { \zref@newlabel }
                {
%    \end{macrocode}
% Found a \cs{zref@label}, signal it.
%    \begin{macrocode}
                  \bool_set_true:N \l_tmpa_bool
                }
                {
                  \bool_if:NTF \l_tmpa_bool
                    {
                      \bool_set_false:N \l_tmpa_bool
                      \int_incr:N \l_tmpa_int
                      \prop_gput:Nxx \g_@@_auxfile_lblseq_prop
                        { \l_tmpb_tl } { \int_use:N \l_tmpa_int }
                    }
                    {
%    \end{macrocode}
% If there is not a match of the first token with \cs{zref@newlabel}, break
% the loop and discard the rest of the line, to ensure no babel calls to
% \cs{catcode} in the \file{.aux} file get expanded.  This also breaks the
% loop and discards the rest of the \cs{zref@newlabel} lines after we got the
% label we wanted, since we reset \cs{l_tmpa_bool} in the \texttt{T} branch.
%    \begin{macrocode}
                      \tl_map_break:
                    }
                }
            }
        }
    \group_end:
    \ior_close:N \g_tmpa_ior
  }
%    \end{macrocode}
%
%
%
% The alternate method I had considered (more than that...) for this was using
% yx coordinates supplied by \pkg{zref}'s \pkg{savepos} module.  However, this
% approach brought in a number of complexities, including the need to patch
% either \cs{zref@label} or \cs{ZREF@label}.  In addition, the technique was
% at the bottom fundamentally flawed.  Ulrike Fischer was very much right when
% she said that ``structure and position are two different beasts''
% (\url{https://github.com/ho-tex/zref/issues/12#issuecomment-880022576}).  It
% is true that the checks based on it behaved decently, in normal
% circumstances, and except for outrageous label placement by the user, it
% would return the expected results.  We don't really need exact coordinates
% to decide ``above/below''.  Besides, it would do an exact job for the
% dedicated target macros of this package.  It is also true that the ``page''
% for \cs{pageref} is stored with the value of where the \cs{label} is placed,
% wherever that may be.  However, I could not conceive a situation where the
% \texttt{yx} criterion would perform clearly better than the
% \texttt{labelseq} one.  And, if that's the case, and considering the
% complications it brings, this check was a slippery slope.  All in all, I've
% decided to drop it.
%
%
%
% \subsection{Counter}
%
% We need a dedicated counter for the labels generated by the checks and
% targets.  The value of the counter is not relevant, we just need it to be
% able to set proper anchors with \cs{refstepcounter}.  And, since I couldn't
% find a \cs{refstepcounter} equivalent in L3, we use a standard 2e counter
% here.  I'm also using the technique to ensure the counter is never reset
% that is used by \file{zref-abspage.sty} and \cs{zref@require@unique}.
% Indeed, the requirements are the same, we need numbers ensured to be
% \emph{unique} in the counter.
%
%    \begin{macrocode}
\begingroup
  \let \@addtoreset \ltx@gobbletwo
  \newcounter { zrefcheck }
\endgroup
\setcounter { zrefcheck } { 0 }
%    \end{macrocode}
%
%
%
% \subsection{Label formats}
%
% \begin{macro}{\@@_check_lblfmt:n}
%   \begin{syntax}
%     \cs{@@_check_lblfmt:n} \Arg{check id int}
%   \end{syntax}
%    \begin{macrocode}
\cs_new:Npn \@@_check_lblfmt:n #1 { zrefcheck@ \int_use:N #1 }
%    \end{macrocode}
% \end{macro}
%
% \begin{macro}{\@@_end_lblfmt:n}
%   \begin{syntax}
%     \cs{@@_end_lblfmt:n} \Arg{label}
%   \end{syntax}
%    \begin{macrocode}
\cs_new:Npn \@@_end_lblfmt:n #1 { #1 @zrefcheck }
%    \end{macrocode}
% \end{macro}
%
%
%
% \subsection{Property values}
%
%
% \begin{macro}[int]{\zrefcheck_get_astl:nnn}
%   A convenience function to retrieve property values from labels.  Uses
%   \cs{g_@@_auxfile_lblseq_prop} for \texttt{lblseq}, and calls
%   \cs{zref@extractdefault} for everything else.
%
%   We cannot use the ``return value'' of \cs{@@_get_astl:nnn} or
%   \cs{@@_get_asint:nnn} directly, because we need to use the retrieved
%   property values as arguments in the checks, however we use here a number
%   of non-expandable operations.  Hence, we receive a local \texttt{tl/int}
%   variable as third argument and set that, so that it is available (and
%   expandable) at the place of use, and also make these functions `protected'
%   (see egreg's \url{https://tex.stackexchange.com/a/572903}: ``a function
%   that performs assignments should be \texttt{protected}'').  For this
%   reason, we do not group here, because we are passing a local variable
%   around, but it is expected this function will be called within a group.
%
%   We're returning \cs{c_empty_tl} in case of failure to find the intended
%   property value (explicitly in \cs{zref@extractdefault}, but that is also
%   what \cs{tl_clear:N} does).
%
%   \begin{syntax}
%     \cs{zrefcheck_get_astl:nnn} \Arg{label} \Arg{prop} \Arg{tl var}
%   \end{syntax}
%    \begin{macrocode}
\cs_new_protected:Npn \zrefcheck_get_astl:nnn #1#2#3
  {
    \tl_clear:N #3
    \tl_if_eq:nnTF {#2} { lblseq }
      {
        \prop_get:NnNF \g_@@_auxfile_lblseq_prop {#1} #3
          {
            \msg_warning:nnnn { zref-check }
              { property-not-in-label } {#1} {#2}
          }
      }
      {
%    \end{macrocode}
% There are three things we need to check to ensure the information we are
% trying to retrieve here exists: the existence of \Arg{label}, the existence
% of \Arg{prop}, and whether the particular label being queried actually
% contains the property.  If that's all in place, the value is passed to the
% checks, and it's their responsibility to verify the consistency of this
% value.
%
% The existence of the label is an user facing issue, and a warning for this
% is placed in \cs{@@_zcheck:nnnnn} (and done with \cs{zref@refused}).  We do
% check here though for definition with \cs{zref@ifrefundefined} and silently
% do nothing if it is undefined, to reduce irrelevant warnings in a fresh
% compilation round.  The other two are more ``internal'' problems, either
% some problem with the checks, or with the configuration of \pkg{zref} for
% their consumption.
%    \begin{macrocode}
        \zref@ifrefundefined {#1}
          {}
          {
            \zref@ifpropundefined {#2}
              { \msg_warning:nnnn { zref-check } { property-undefined } {#2} }
              {
                \zref@ifrefcontainsprop {#1} {#2}
                  {
                    \tl_set:Nx #3
                      { \zref@extractdefault {#1} {#2} { \c_empty_tl } }
                  }
                  {
                    \msg_warning:nnnn
                      { zref-check } { property-not-in-label } {#1} {#2}
                  }
              }
          }
      }
  }
%    \end{macrocode}
% \end{macro}
%
%
% \begin{variable}{\l_@@_integer_bool}
%   \cs{zrefcheck_get_asint:nnn} is a very convenient wrapper around the more
%   general \cs{zrefcheck_get_astl:nnn}, since almost always we'll be wanting
%   to compare numbers in the checks.  However, it is quite hard for it to
%   ensure an integer is \emph{always} returned in the case of errors.  And
%   those do occur, even in a well structured document (e.g., in a first round
%   of compilation).  To complicate things, the L3 integer predicates are
%   \emph{very} sensitive to receiving any other kind of data, and they
%   \emph{scream}.  To handle this \cs{zrefcheck_get_asint:nnn} uses
%   \cs{l_@@_integer_bool} to signal if an integer could not be returned.  To
%   use this function always set \cs{l_@@_integer_bool} to true first, then
%   call it as much as you need.  If any of these calls got is returning
%   anything which is not an integer, \cs{l_@@_integer_bool} will have been
%   set to false, and you should check that this hasn't happened before
%   actually comparing the integers (\cs{bool_lazy_and:nnTF} is your friend).
%    \begin{macrocode}
\bool_new:N \l_@@_integer_bool
%    \end{macrocode}
% \end{variable}
%
% \begin{variable}{\l_@@_propval_tl}
%    \begin{macrocode}
\tl_new:N \l_@@_propval_tl
%    \end{macrocode}
% \end{variable}
%
% \begin{macro}[int]{\zrefcheck_get_asint:nnn}
%   \begin{syntax}
%     \cs{zrefcheck_get_asint:nnn} \Arg{label} \Arg{prop} \Arg{int var}
%   \end{syntax}
%    \begin{macrocode}
\cs_new_protected:Npn \zrefcheck_get_asint:nnn #1#2#3
  {
    \zrefcheck_get_astl:nnn {#1} {#2} { \l_@@_propval_tl }
    \@@_is_integer:nTF { \l_@@_propval_tl }
      {
%    \end{macrocode}
% Make it an integer data type.
%    \begin{macrocode}
        \int_set:Nn #3 { \int_eval:n { \l_@@_propval_tl } }
      }
      {
        \bool_set_false:N \l_@@_integer_bool
        \zref@ifrefundefined {#1}
%    \end{macrocode}
% Keep silent if ref is undefined to reduce irrelevant warnings in a fresh
% compilation round.  Again, this is also not the point to check for undefined
% references, that's a task for \cs{@@_zcheck:nnnnn}.
%    \begin{macrocode}
          { }
          {
            \msg_warning:nnnn { zref-check }
              { property-not-integer } {#2} {#1}
          }
      }
  }
%    \end{macrocode}
% \end{macro}
%
%
%
%
% \section{User interface}
%
%
% \subsection{\cs{zcheck}}
%
% \begin{macro}[int]{\zcheck}
%   The \marg{text} argument of \cs{zcheck} should not be long, since
%   \cs{hyperlink} cannot receive a long argument.  Besides, there is no
%   reason for it to be.  Note, also, that hyperlinks crossing page boundaries
%   have some known issues: \url{https://tex.stackexchange.com/a/182769},
%   \url{https://tex.stackexchange.com/a/54607},
%   \url{https://tex.stackexchange.com/a/179907}.
%
%   \begin{syntax}
%     \cs{zcheck}\meta{*}\oarg{checks/options}\marg{labels}\marg{text}
%   \end{syntax}
%
%    \begin{macrocode}
\NewDocumentCommand \zcheck
  { s O { } > { \SplitList { , } } m m }
  { \zref@wrapper@babel \@@_zcheck:nnnn {#3} {#1} {#2} {#4} }
%    \end{macrocode}
% \end{macro}
%
%
% \begin{variable}
%   {
%     \g_@@_id_int ,
%     \l_@@_checkbeg_tl ,
%     \l_@@_checkend_tl ,
%     \l_@@_link_label_tl ,
%     \l_@@_link_anchor_tl ,
%     \l_@@_link_star_bool
%   }
%    \begin{macrocode}
\int_new:N \g_@@_id_int
\tl_new:N \l_@@_checkbeg_tl
\tl_new:N \l_@@_checkend_tl
\tl_new:N \l_@@_link_label_tl
\tl_new:N \l_@@_link_anchor_tl
\bool_new:N \l_@@_link_star_bool
%    \end{macrocode}
% \end{variable}
%
% \begin{macro}{\@@_zcheck:nnnn}
%   An intermediate internal function, which does the actual heavy lifting,
%   and places \Arg{labels} as first argument, so that it can be protected by
%   \cs{zref@wrapper@babel} in \cs{zcheck}.  This is the same procedure as the
%   one used in the definition of \cs{zref} in \file{zref-user.sty} for
%   protection of \pkg{babel} active characters.
%
%   \begin{syntax}
%     \cs{@@_zcheck:nnnn} \Arg{labels} \Arg{*} \Arg{checks/options} \Arg{text}
%   \end{syntax}
%
%    \begin{macrocode}
\cs_new_protected:Npn \@@_zcheck:nnnn #1#2#3#4
  {
    \group_begin:
%    \end{macrocode}
% Process local options and checks.
%    \begin{macrocode}
      \keys_set:nn { zref-check / zcheck } {#3}
%    \end{macrocode}
% Names of the labels for this zcheck call.
%    \begin{macrocode}
      \int_gincr:N \g_@@_id_int
      \tl_set:Nx \l_@@_checkbeg_tl
        { \@@_check_lblfmt:n { \g_@@_id_int } }
      \tl_set:Nx \l_@@_checkend_tl
        { \@@_end_lblfmt:n { \l_@@_checkbeg_tl } }
%    \end{macrocode}
% Set checkbeg label.
%    \begin{macrocode}
      \zref@labelbylist { \l_@@_checkbeg_tl } { zrefcheck }
%    \end{macrocode}
% Typeset \marg{text}, with hyperlink when appropriate.  Even though the first
% argument can receive a list of labels, there is no meaningful way to set
% links to multiple targets.  Hence, only the first one is considered for
% hyperlinking.
%    \begin{macrocode}
      \tl_set:Nn \l_@@_link_label_tl { \tl_head:n {#1} }
      \bool_set:Nn \l_@@_link_star_bool {#2}
      \zref@ifrefundefined { \l_@@_link_label_tl }
%    \end{macrocode}
% If the reference is undefined, just typeset.
%    \begin{macrocode}
        {#4}
        {
          \bool_if:nTF
            {
              \l_@@_use_hyperref_bool &&
              ! \l_@@_link_star_bool
            }
            {
              \exp_args:Nx \zrefcheck_get_astl:nnn
                { \l_@@_link_label_tl }
                { anchor } { \l_@@_link_anchor_tl }
              \hyperlink { \l_@@_link_anchor_tl } {#4}
            }
            {#4}
        }
%    \end{macrocode}
% Set checkend label.
%    \begin{macrocode}
      \bool_if:NT \l_@@_zcheck_end_label_bool
        { \zref@labelbylist { \l_@@_checkend_tl } { zrefcheck } }
%    \end{macrocode}
% Check if \meta{labels} are defined.
%    \begin{macrocode}
      \tl_map_function:nN {#1} \zref@refused
%    \end{macrocode}
% Run the checks.
%    \begin{macrocode}
      \@@_run_checks:nnv
        { \l_@@_zcheck_checks_seq } {#1} { l_@@_checkbeg_tl }
    \group_end:
  }
%    \end{macrocode}
% \end{macro}
%
%
% \subsection{Targets}
%
% \begin{macro}[int]{\zctarget}
%   \begin{syntax}
%     \cs{zctarget}\marg{label}\marg{text}
%   \end{syntax}
%    \begin{macrocode}
\NewDocumentCommand \zctarget { m +m }
  {
%    \end{macrocode}
% Group contents of \cs{zctarget} to avoid leaking the effects of
% \cs{refstepcounter} over \cs{@currentlabel}.  The same care is not needed
% for \texttt{zcregion}, since the environment is already grouped.
%    \begin{macrocode}
    \group_begin:
    \refstepcounter { zrefcheck }
    \zref@wrapper@babel \@@_target_label:n {#1}
    #2
    \zref@wrapper@babel
      \zref@labelbylist { \@@_end_lblfmt:n {#1} } { zrefcheck }
    \group_end:
  }
%    \end{macrocode}
% \end{macro}
%
% \begin{macro}[int]{zcregion}
%   \begin{syntax}
%     |\begin{zcregion}|\marg{label}
%       |  ...|
%     |\end{zcregion}|
%   \end{syntax}
%    \begin{macrocode}
\NewDocumentEnvironment {zcregion} { m }
  {
    \refstepcounter { zrefcheck }
    \zref@wrapper@babel \@@_target_label:n {#1}
  }
  {
    \zref@wrapper@babel
      \zref@labelbylist { \@@_end_lblfmt:n {#1} } { zrefcheck }
  }
%    \end{macrocode}
% \end{macro}
%
%
%
% \section{Checks}
%
% What is needed define a \pkg{zref-check} check?
%
% First, a conditional function defined with:
%
% \cs{prg_new_protected_conditional:Npnn} \cs{@@_check_\meta{check}:nn} |#1#2 { F }|
%
% \noindent where \meta{check} is the name of the check, the first argument is
% the \Arg{label} and the second the \Arg{reference}.  The existence of the
% check is verified by the existence of the function with this name-scheme
% (and signatures).  As usual, this function must return either
% \cs{prg_return_true:} or \cs{prg_return_false:}.  Of course, you can define
% other variants if you need them internally, it is just that what the package
% does expect and verifies is the existence of the \texttt{:nnF} variant.
%
% Note that the naming convention of the checks adopts the perspective of the
% \meta{reference}.  That is, the ``before'' check should return true if the
% \meta{label} occurs before the ``reference''.
%
% The check conditionals are expected to retrieve \pkg{zref}'s label
% information with \cs{zrefcheck_get_astl:nnn} or
% \cs{zrefcheck_get_asint:nnn}.  Also, technically speaking, the
% \meta{reference} argument is also a label, actually a pair of them, as set
% by \cs{zcheck}.  For the ``labels'', any \pkg{zref} property in \pkg{zref}'s
% main list is available, the ``references'' store the properties in the
% \texttt{zrefcheck} list.  Besides those, there is also the \texttt{lblseq}
% (fake) property (for either ``labels'' or ``references''), stored in
% \cs{g_@@_auxfile_lblseq_prop}.
%
% Second, the required properties of labels and references must be duly
% registered for \pkg{zref}.  This can be done with \cs{zref@newprop},
% \cs{zref@addprop} and friends, as usual.
%
% Third, the check must be registered as a key which gets setup in
% \cs{zcheck} by the \texttt{ zref-check / zcheck } key set.
%
% Fourth, if the check requires only a single label to work, it should be
% registered in \{c_@@_single_label_checks_seq}.
%
%
% \subsection{Single label checks}
%
%
% Some checks do not require an ``end label'' in \cs{zcheck}, notably the
% sectioning ones, which don't rely on page boundaries.  Hence, in case
% \cs{zcheck} only calls checks in this set, we can spare the setting of the
% end label.
%
% \begin{variable}{\c_@@_single_label_checks_seq}
%    \begin{macrocode}
\seq_new:N \c_@@_single_label_checks_seq
\seq_set_from_clist:Nn \c_@@_single_label_checks_seq
  {
    thischap ,
    prevchap ,
    nextchap ,
    chapsbefore ,
    chapsafter ,
    thissec ,
    prevsec ,
    nextsec ,
    secsbefore ,
    secsafter ,
  }
%    \end{macrocode}
% \end{variable}
%
%
%
% \subsection{Setup}
%
% \begin{variable}{\l_@@_zcheck_checks_seq,\l_@@_end_label_required_bool}
%    \begin{macrocode}
\seq_new:N \l_@@_zcheck_checks_seq
\bool_new:N \l_@@_zcheck_end_label_bool
%    \end{macrocode}
% \end{variable}
%
%
% First, we inherit all the main options into the keys of \texttt{zref-check /
% zcheck}.
%    \begin{macrocode}
\keys_define:nn { } { zref-check / zcheck .inherit:n = zref-check }
%    \end{macrocode}
%
% Then we add the checks to it.
%    \begin{macrocode}
\clist_map_inline:nn
  {
    thispage ,
    prevpage ,
    nextpage ,
    facing ,
    above ,
    below ,
    pagesbefore ,
    ppbefore ,
    pagesafter ,
    ppafter ,
    before ,
    after ,
    thischap ,
    prevchap ,
    nextchap ,
    chapsbefore ,
    chapsafter ,
    thissec ,
    prevsec ,
    nextsec ,
    secsbefore ,
    secsafter ,
    close ,
    far ,
  }
  {
    \keys_define:nn { zref-check / zcheck }
      {
        #1 .code:n =
          {
            \seq_put_right:Nn \l_@@_zcheck_checks_seq {#1}
            \seq_if_in:NnF \c_@@_single_label_checks_seq {#1}
              { \bool_set_true:N \l_@@_zcheck_end_label_bool }
          } ,
        #1 .value_forbidden:n = true ,
     }
  }
%    \end{macrocode}
%
%
%
% \subsection{Running}
%
% \begin{macro}{\@@_run_checks:nnn, \@@_run_checks:nnv}
%   \begin{syntax}
%     \cs{@@_run_checks:nnn} \Arg{checks} \Arg{labels} \Arg{reference}
%   \end{syntax}
%   \meta{checks} are expected to be received as a sequence variable.
%    \begin{macrocode}
\cs_new_protected:Npn \@@_run_checks:nnn #1#2#3
  {
    \group_begin:
      \tl_map_inline:nn {#2}
        {
          \seq_map_inline:Nn #1
            { \@@_do_check:nnn {####1} {##1} {#3} }
        }
    \group_end:
  }
\cs_generate_variant:Nn \@@_run_checks:nnn { nnv }
%    \end{macrocode}
% \end{macro}
%
%
% \begin{variable}
%   {
%     \l_@@_passedcheck_bool ,
%     \l_@@_onpage_bool ,
%     \c_@@_onpage_checks_seq
%   }
%    \begin{macrocode}
\bool_new:N \l_@@_passedcheck_bool
\bool_new:N \l_@@_onpage_bool
\seq_new:N \c_@@_onpage_checks_seq
\seq_set_from_clist:Nn \c_@@_onpage_checks_seq
  { above , below , before , after }
%    \end{macrocode}
% \end{variable}
%
%
% Variant not provided by \pkg{expl3}.
%    \begin{macrocode}
\cs_generate_variant:Nn \exp_args:Nnno { Nnoo }
%    \end{macrocode}
%
% \begin{macro}{\@@_do_check:nnn}
%   \begin{syntax}
%     \cs{@@_do_check:nnn} \Arg{check} \Arg{label beg} \Arg{reference beg}
%   \end{syntax}
%    \begin{macrocode}
\cs_new_protected:Npn \@@_do_check:nnn #1#2#3
  {
    \group_begin:
%    \end{macrocode}
% \meta{label beg} may be defined or not, it is arbitrary user input.  Whether
% this is the case is checked in \cs{@@_zcheck:nnnnn}, and due warning already
% ensues.  And there is no point in checking ``relative position'' of an
% undefined label.  Hence, in the absence of |#2|, we do nothing at all here.
%    \begin{macrocode}
      \zref@ifrefundefined {#2}
        {}
        {
          \tl_if_empty:nF {#1}
            {
              \bool_set_true:N \l_@@_passedcheck_bool
              \bool_set_false:N \l_@@_onpage_bool
              \cs_if_exist:cTF { @@_check_ #1 :nnF }
                {
                  % ``label beg'' vs ``reference beg''.
                  \use:c { @@_check_ #1 :nnF }
                    {#2} {#3}
                    { \bool_set_false:N \l_@@_passedcheck_bool }
                  % ``reference end'' \emph{may} exist or not depending on the
                  % checks.
                  \zref@ifrefundefined { \@@_end_lblfmt:n {#3} }
                    {
                      % ``label end'' \emph{may} have been created by the
                      % target commands.
                      \zref@ifrefundefined { \@@_end_lblfmt:n {#2} }
                        {}
                        {
                          % ``label end'' vs ``reference beg''.
                          \exp_args:Nno \use:c { @@_check_ #1 :nnF }
                            { \@@_end_lblfmt:n {#2} } {#3}
                            { \bool_set_false:N \l_@@_passedcheck_bool }
                        }
                    }
                    {
                      % ``label beg'' vs ``reference end''.
                      \exp_args:Nnno \use:c { @@_check_ #1 :nnF }
                        {#2} { \@@_end_lblfmt:n {#3} }
                        { \bool_set_false:N \l_@@_passedcheck_bool }
                      % ``label end'' \emph{may} have been created by the
                      % target commands.
                      \zref@ifrefundefined { \@@_end_lblfmt:n {#2} }
                        {}
                        {
                          % ``label end'' vs ``reference beg''.
                          \exp_args:Nno \use:c { @@_check_ #1 :nnF }
                            { \@@_end_lblfmt:n {#2} } {#3}
                            { \bool_set_false:N \l_@@_passedcheck_bool }
                          % ``label end'' vs ``reference end''.
                          \exp_args:Nnoo \use:c { @@_check_ #1 :nnF }
                            { \@@_end_lblfmt:n {#2} }
                            { \@@_end_lblfmt:n {#3} }
                            { \bool_set_false:N \l_@@_passedcheck_bool }
                        }
                    }
%    \end{macrocode}
% Handle option \opt{onpage=msg}.  This is only granted for tests which
% perform ``within this page'' checks (\opt{above}, \opt{below}, \opt{before},
% \opt{after}) \emph{and} if any of the two by two checks uses a ``within this
% page'' comparison.  If both conditions are met, signal.
%    \begin{macrocode}
                  \seq_if_in:NnT \c_@@_onpage_checks_seq {#1}
                    {
                      \@@_check_thispage:nnT
                        {#2} {#3}
                        { \bool_set_true:N \l_@@_onpage_bool }
                      \@@_check_thispage:nnT
                        {#2} { \@@_end_lblfmt:n {#3} }
                        { \bool_set_true:N \l_@@_onpage_bool }
                      \zref@ifrefundefined { \@@_end_lblfmt:n {#2} }
                        {}
                        {
                          \@@_check_thispage:nnT
                            { \@@_end_lblfmt:n {#2} } {#3}
                            { \bool_set_true:N \l_@@_onpage_bool }
                          \@@_check_thispage:nnT
                            { \@@_end_lblfmt:n {#2} }
                            { \@@_end_lblfmt:n {#3} }
                            { \bool_set_true:N \l_@@_onpage_bool }
                        }
                    }
                  \bool_if:NTF \l_@@_passedcheck_bool
                    {
                      \bool_if:nT
                        {
                          \l_@@_msgonpage_bool &&
                          \l_@@_onpage_bool
                        }
                        {
                          \@@_message:nnnx { double-check } {#1} {#2}
                            { \zref@extractdefault {#3} {page} {'unknown'} }
                        }
                    }
                    {
                      \@@_message:nnnx { check-failed } {#1} {#2}
                        { \zref@extractdefault {#3} {page} {'unknown'} }
                    }
                }
                { \msg_warning:nnn { zref-check } { check-missing } {#1} }
            }
        }
    \group_end:
  }
%    \end{macrocode}
% \end{macro}
%
%
% \subsection{Conditionals}
%
% \begin{variable}
%   {
%     \l_@@_lbl_int ,
%     \l_@@_ref_int ,
%     \l_@@_lbl_b_int ,
%     \l_@@_ref_b_int
%   }
%   More readable scratch variables for the tests.
%    \begin{macrocode}
\int_new:N \l_@@_lbl_int
\int_new:N \l_@@_ref_int
\int_new:N \l_@@_lbl_b_int
\int_new:N \l_@@_ref_b_int
%    \end{macrocode}
% \end{variable}
%
%
% \subsubsection{This page}
%
% \begin{macro}{\@@_check_thispage:nn}
%    \begin{macrocode}
\prg_new_protected_conditional:Npnn \@@_check_thispage:nn #1#2 { T , F , TF }
  {
    \group_begin:
      \bool_set_true:N \l_@@_integer_bool
      \zrefcheck_get_asint:nnn {#1} { abspage } { \l_@@_lbl_int }
      \zrefcheck_get_asint:nnn {#2} { abspage } { \l_@@_ref_int }
      \bool_lazy_and:nnTF
        { \l_@@_integer_bool }
        {
          \int_compare_p:nNn
            { \l_@@_lbl_int } = { \l_@@_ref_int } &&
%    \end{macrocode}
% `0' is the default value of \texttt{abspage}, but this value should not
% happen normally for this property, since even the first page, after it gets
% shipped out, will receive value `1'.  So, if we do find `0' here, better
% signal something is wrong.  This comment extends to all page number checks.
%    \begin{macrocode}
            ! \int_compare_p:nNn { \l_@@_lbl_int } = { 0 } &&
            ! \int_compare_p:nNn { \l_@@_ref_int } = { 0 }
        }
        { \group_insert_after:N \prg_return_true:  }
        { \group_insert_after:N \prg_return_false: }
    \group_end:
  }
%    \end{macrocode}
% \end{macro}
%
%
% \subsubsection{On page}
%
% \begin{macro}{\@@_check_above:nn, \@@_check_below:nn}
%    \begin{macrocode}
\prg_new_protected_conditional:Npnn \@@_check_above:nn #1#2 { F , TF }
  {
    \group_begin:
      \@@_check_thispage:nnTF {#1} {#2}
        {
          \bool_set_true:N \l_@@_integer_bool
          \zrefcheck_get_asint:nnn {#1} { lblseq } { \l_@@_lbl_int }
          \zrefcheck_get_asint:nnn {#2} { lblseq } { \l_@@_ref_int }
          \bool_lazy_and:nnTF
            { \l_@@_integer_bool }
            {
              \int_compare_p:nNn
                { \l_@@_lbl_int } < { \l_@@_ref_int } &&
              ! \int_compare_p:nNn { \l_@@_lbl_int } = { 0 } &&
              ! \int_compare_p:nNn { \l_@@_ref_int } = { 0 }
            }
            { \group_insert_after:N \prg_return_true:  }
            { \group_insert_after:N \prg_return_false: }
        }
        { \group_insert_after:N \prg_return_false: }
    \group_end:
  }
\prg_new_protected_conditional:Npnn \@@_check_below:nn #1#2 { F , TF }
  {
    \@@_check_thispage:nnTF {#1} {#2}
      {
        \@@_check_above:nnTF {#1} {#2}
          { \prg_return_false: }
          { \prg_return_true:  }
      }
      { \prg_return_false: }
  }
%    \end{macrocode}
% \end{macro}
%
%
% \subsubsection{Before / After}
%
% \begin{macro}{\@@_check_before:nn, \@@_check_after:nn}
%    \begin{macrocode}
\prg_new_protected_conditional:Npnn \@@_check_before:nn #1#2 { F }
  {
    \@@_check_pagesbefore:nnTF {#1} {#2}
      { \prg_return_true: }
      {
        \@@_check_above:nnTF {#1} {#2}
          { \prg_return_true:  }
          { \prg_return_false: }
      }
  }
\prg_new_protected_conditional:Npnn \@@_check_after:nn #1#2 { F }
  {
    \@@_check_pagesafter:nnTF {#1} {#2}
      { \prg_return_true: }
      {
        \@@_check_below:nnTF {#1} {#2}
          { \prg_return_true:  }
          { \prg_return_false: }
      }
  }
%    \end{macrocode}
% \end{macro}
%
%
% \subsubsection{Pages}
%
% \begin{macro}
%   {
%     \@@_check_nextpage:nn ,
%     \@@_check_prevpage:nn ,
%     \@@_check_pagesbefore:nn ,
%     \@@_check_ppbefore:nn ,
%     \@@_check_pagesafter:nn ,
%     \@@_check_ppafter:nn ,
%     \@@_check_facing:nn
%   }
%    \begin{macrocode}
\prg_new_protected_conditional:Npnn \@@_check_nextpage:nn #1#2 { F }
  {
    \group_begin:
      \bool_set_true:N \l_@@_integer_bool
      \zrefcheck_get_asint:nnn {#1} { abspage } { \l_@@_lbl_int }
      \zrefcheck_get_asint:nnn {#2} { abspage } { \l_@@_ref_int }
      \bool_lazy_and:nnTF
        { \l_@@_integer_bool }
        {
          \int_compare_p:nNn
            { \l_@@_lbl_int } = { \l_@@_ref_int + 1 } &&
          ! \int_compare_p:nNn { \l_@@_lbl_int } = { 0 } &&
          ! \int_compare_p:nNn { \l_@@_ref_int } = { 0 }
        }
        { \group_insert_after:N \prg_return_true:  }
        { \group_insert_after:N \prg_return_false: }
    \group_end:
  }
\prg_new_protected_conditional:Npnn \@@_check_prevpage:nn #1#2 { F }
  {
    \group_begin:
      \bool_set_true:N \l_@@_integer_bool
      \zrefcheck_get_asint:nnn {#1} { abspage } { \l_@@_lbl_int }
      \zrefcheck_get_asint:nnn {#2} { abspage } { \l_@@_ref_int }
      \bool_lazy_and:nnTF
        { \l_@@_integer_bool }
        {
          \int_compare_p:nNn
            { \l_@@_lbl_int } = { \l_@@_ref_int - 1 } &&
          ! \int_compare_p:nNn { \l_@@_lbl_int } = { 0 } &&
          ! \int_compare_p:nNn { \l_@@_ref_int } = { 0 }
        }
        { \group_insert_after:N \prg_return_true:  }
        { \group_insert_after:N \prg_return_false: }
    \group_end:
  }
\prg_new_protected_conditional:Npnn \@@_check_pagesbefore:nn #1#2 { F , TF }
  {
    \group_begin:
      \bool_set_true:N \l_@@_integer_bool
      \zrefcheck_get_asint:nnn {#1} { abspage } { \l_@@_lbl_int }
      \zrefcheck_get_asint:nnn {#2} { abspage } { \l_@@_ref_int }
      \bool_lazy_and:nnTF
        { \l_@@_integer_bool }
        {
          \int_compare_p:nNn
            { \l_@@_lbl_int } < { \l_@@_ref_int } &&
          ! \int_compare_p:nNn { \l_@@_lbl_int } = { 0 } &&
          ! \int_compare_p:nNn { \l_@@_ref_int } = { 0 }
        }
        { \group_insert_after:N \prg_return_true:  }
        { \group_insert_after:N \prg_return_false: }
    \group_end:
  }
\cs_new_eq:NN \@@_check_ppbefore:nnF \@@_check_pagesbefore:nnF
\prg_new_protected_conditional:Npnn \@@_check_pagesafter:nn #1#2 { F , TF }
  {
    \group_begin:
      \bool_set_true:N \l_@@_integer_bool
      \zrefcheck_get_asint:nnn {#1} { abspage } { \l_@@_lbl_int }
      \zrefcheck_get_asint:nnn {#2} { abspage } { \l_@@_ref_int }
      \bool_lazy_and:nnTF
        { \l_@@_integer_bool }
        {
          \int_compare_p:nNn
            { \l_@@_lbl_int } > { \l_@@_ref_int } &&
          ! \int_compare_p:nNn { \l_@@_lbl_int } = { 0 } &&
          ! \int_compare_p:nNn { \l_@@_ref_int } = { 0 }
        }
        { \group_insert_after:N \prg_return_true:  }
        { \group_insert_after:N \prg_return_false: }
    \group_end:
  }
\cs_new_eq:NN \@@_check_ppafter:nnF \@@_check_pagesafter:nnF
\prg_new_protected_conditional:Npnn \@@_check_facing:nn #1#2 { F }
  {
    \group_begin:
      \bool_set_true:N \l_@@_integer_bool
      \zrefcheck_get_asint:nnn {#1} { abspage } { \l_@@_lbl_int }
      \zrefcheck_get_asint:nnn {#2} { abspage } { \l_@@_ref_int }
      \bool_lazy_and:nnTF
        { \l_@@_integer_bool }
        {
%    \end{macrocode}
% There exists no ``facing'' page if the document is not twoside.
%    \begin{macrocode}
          \legacy_if_p:n { @twoside } &&
%    \end{macrocode}
% Now we test ``facing''.
%    \begin{macrocode}
          (
            (
              \int_if_odd_p:n { \l_@@_ref_int } &&
              \int_compare_p:nNn
                { \l_@@_lbl_int } = { \l_@@_ref_int - 1 }
            ) ||
            (
              \int_if_even_p:n { \l_@@_ref_int } &&
              \int_compare_p:nNn
                { \l_@@_lbl_int } = { \l_@@_ref_int + 1 }
            )
          ) &&
          ! \int_compare_p:nNn { \l_@@_lbl_int } = { 0 } &&
          ! \int_compare_p:nNn { \l_@@_ref_int } = { 0 }
        }
        { \group_insert_after:N \prg_return_true:  }
        { \group_insert_after:N \prg_return_false: }
    \group_end:
  }
%    \end{macrocode}
% \end{macro}
%
%
% \subsubsection{Close / Far}
%
% \begin{macro}{\@@_check_close:nn, \@@_check_far:nn}
%    \begin{macrocode}
\prg_new_protected_conditional:Npnn \@@_check_close:nn #1#2 { F , TF }
  {
    \group_begin:
      \bool_set_true:N \l_@@_integer_bool
      \zrefcheck_get_asint:nnn {#1} { abspage } { \l_@@_lbl_int }
      \zrefcheck_get_asint:nnn {#2} { abspage } { \l_@@_ref_int }
      \bool_lazy_and:nnTF
        { \l_@@_integer_bool }
        {
          \int_compare_p:nNn
            { \int_abs:n { \l_@@_lbl_int - \l_@@_ref_int } }
            <
            { \l_@@_close_range_int + 1 } &&
          ! \int_compare_p:nNn { \l_@@_lbl_int } = { 0 } &&
          ! \int_compare_p:nNn { \l_@@_ref_int } = { 0 }
        }
        { \group_insert_after:N \prg_return_true:  }
        { \group_insert_after:N \prg_return_false: }
    \group_end:
  }
\prg_new_protected_conditional:Npnn \@@_check_far:nn #1#2 { F }
  {
    \@@_check_close:nnTF {#1} {#2}
      { \prg_return_false: }
      { \prg_return_true:  }
  }
%    \end{macrocode}
% \end{macro}
%
%
% \subsubsection{Chapter}
%
% \begin{macro}
%   {
%     \@@_check_thischap:nn ,
%     \@@_check_nextchap:nn ,
%     \@@_check_prevchap:nn ,
%     \@@_check_chapsafter:nn ,
%     \@@_check_chapsbefore:nn
%   }
%    \begin{macrocode}
\prg_new_protected_conditional:Npnn \@@_check_thischap:nn #1#2 { F }
  {
    \group_begin:
      \bool_set_true:N \l_@@_integer_bool
      \zrefcheck_get_asint:nnn {#1} { zc@abschap } { \l_@@_lbl_int }
      \zrefcheck_get_asint:nnn {#2} { zc@abschap } { \l_@@_ref_int }
      \bool_lazy_and:nnTF
        { \l_@@_integer_bool }
        {
          \int_compare_p:nNn
            { \l_@@_lbl_int } = { \l_@@_ref_int } &&
%    \end{macrocode}
% `0' is the default value of \texttt{zc@abschap} property, and means here no
% \cs{chapter} has yet been issued, therefore it cannot be ``this chapter'',
% nor ``the next chapter'', nor ``the previous chapter'', it is just ``no
% chapter''.  Note, however, that a statement about a ``future'' chapter does
% not require the ``current'' one to exist.  This comment extends to all
% chapter checks.
%    \begin{macrocode}
          ! \int_compare_p:nNn { \l_@@_lbl_int } = { 0 } &&
          ! \int_compare_p:nNn { \l_@@_ref_int } = { 0 }
        }
        { \group_insert_after:N \prg_return_true:  }
        { \group_insert_after:N \prg_return_false: }
    \group_end:
  }
\prg_new_protected_conditional:Npnn \@@_check_nextchap:nn #1#2 { F }
  {
    \group_begin:
      \bool_set_true:N \l_@@_integer_bool
      \zrefcheck_get_asint:nnn {#1} { zc@abschap } { \l_@@_lbl_int }
      \zrefcheck_get_asint:nnn {#2} { zc@abschap } { \l_@@_ref_int }
      \bool_lazy_and:nnTF
        { \l_@@_integer_bool }
        {
          \int_compare_p:nNn
            { \l_@@_lbl_int } = { \l_@@_ref_int + 1 } &&
          ! \int_compare_p:nNn { \l_@@_lbl_int } = { 0 }
        }
        { \group_insert_after:N \prg_return_true:  }
        { \group_insert_after:N \prg_return_false: }
    \group_end:
  }
\prg_new_protected_conditional:Npnn \@@_check_prevchap:nn #1#2 { F }
  {
    \group_begin:
      \bool_set_true:N \l_@@_integer_bool
      \zrefcheck_get_asint:nnn {#1} { zc@abschap } { \l_@@_lbl_int }
      \zrefcheck_get_asint:nnn {#2} { zc@abschap } { \l_@@_ref_int }
      \bool_lazy_and:nnTF
        { \l_@@_integer_bool }
        {
          \int_compare_p:nNn
            { \l_@@_lbl_int } = { \l_@@_ref_int - 1 } &&
          ! \int_compare_p:nNn { \l_@@_lbl_int } = { 0 } &&
          ! \int_compare_p:nNn { \l_@@_ref_int } = { 0 }
        }
        { \group_insert_after:N \prg_return_true:  }
        { \group_insert_after:N \prg_return_false: }
    \group_end:
  }
\prg_new_protected_conditional:Npnn \@@_check_chapsafter:nn #1#2 { F }
  {
    \group_begin:
      \bool_set_true:N \l_@@_integer_bool
      \zrefcheck_get_asint:nnn {#1} { zc@abschap } { \l_@@_lbl_int }
      \zrefcheck_get_asint:nnn {#2} { zc@abschap } { \l_@@_ref_int }
      \bool_lazy_and:nnTF
        { \l_@@_integer_bool }
        {
          \int_compare_p:nNn
            { \l_@@_lbl_int } > { \l_@@_ref_int } &&
          ! \int_compare_p:nNn { \l_@@_lbl_int } = { 0 }
        }
        { \group_insert_after:N \prg_return_true:  }
        { \group_insert_after:N \prg_return_false: }
    \group_end:
  }
\prg_new_protected_conditional:Npnn \@@_check_chapsbefore:nn #1#2 { F }
  {
    \group_begin:
      \bool_set_true:N \l_@@_integer_bool
      \zrefcheck_get_asint:nnn {#1} { zc@abschap } { \l_@@_lbl_int }
      \zrefcheck_get_asint:nnn {#2} { zc@abschap } { \l_@@_ref_int }
      \bool_lazy_and:nnTF
        { \l_@@_integer_bool }
        {
          \int_compare_p:nNn
            { \l_@@_lbl_int } < { \l_@@_ref_int } &&
          ! \int_compare_p:nNn { \l_@@_lbl_int } = { 0 } &&
          ! \int_compare_p:nNn { \l_@@_ref_int } = { 0 }
        }
        { \group_insert_after:N \prg_return_true:  }
        { \group_insert_after:N \prg_return_false: }
    \group_end:
  }
%    \end{macrocode}
% \end{macro}
%
%
% \subsubsection{Section}
%
% \begin{macro}
%   {
%     \@@_check_thissec:nn ,
%     \@@_check_nextsec:nn ,
%     \@@_check_prevsec:nn ,
%     \@@_check_secsafter:nn ,
%     \@@_check_secsbefore:nn
%   }
%    \begin{macrocode}
\prg_new_protected_conditional:Npnn \@@_check_thissec:nn #1#2 { F }
  {
    \group_begin:
      \bool_set_true:N \l_@@_integer_bool
      \zrefcheck_get_asint:nnn {#1} { zc@abssec  } { \l_@@_lbl_int }
      \zrefcheck_get_asint:nnn {#2} { zc@abssec  } { \l_@@_ref_int }
      \zrefcheck_get_asint:nnn {#1} { zc@abschap } { \l_@@_lbl_b_int }
      \zrefcheck_get_asint:nnn {#2} { zc@abschap } { \l_@@_ref_b_int }
      \bool_lazy_and:nnTF
        { \l_@@_integer_bool }
        {
          \int_compare_p:nNn
            { \l_@@_lbl_b_int } = { \l_@@_ref_b_int } &&
          \int_compare_p:nNn
            { \l_@@_lbl_int } = { \l_@@_ref_int } &&
%    \end{macrocode}
% `0' is the default value of \texttt{zc@abssec} property, and means here no
% \cs{section} has yet been issued since its counter has been reset, which
% occurs at the beginning of the document and at every chapter.  Hence, as is
% the case for chapters, `0' is just ``not a section''.  The same observation
% about the need of the ``current'' section to exist to be able to refer to a
% ``future'' one also holds.  This comment extends to all section checks.
%    \begin{macrocode}
          ! \int_compare_p:nNn { \l_@@_lbl_int } = { 0 } &&
          ! \int_compare_p:nNn { \l_@@_ref_int } = { 0 }
        }
        { \group_insert_after:N \prg_return_true:  }
        { \group_insert_after:N \prg_return_false: }
    \group_end:
  }
\prg_new_protected_conditional:Npnn \@@_check_nextsec:nn #1#2 { F }
  {
    \group_begin:
      \bool_set_true:N \l_@@_integer_bool
      \zrefcheck_get_asint:nnn {#1} { zc@abssec  } { \l_@@_lbl_int }
      \zrefcheck_get_asint:nnn {#2} { zc@abssec  } { \l_@@_ref_int }
      \zrefcheck_get_asint:nnn {#1} { zc@abschap } { \l_@@_lbl_b_int }
      \zrefcheck_get_asint:nnn {#2} { zc@abschap } { \l_@@_ref_b_int }
      \bool_lazy_and:nnTF
        { \l_@@_integer_bool }
        {
          \int_compare_p:nNn
            { \l_@@_lbl_b_int } = { \l_@@_ref_b_int } &&
          \int_compare_p:nNn
            { \l_@@_lbl_int } = { \l_@@_ref_int + 1 } &&
          ! \int_compare_p:nNn { \l_@@_lbl_int } = { 0 }
        }
        { \group_insert_after:N \prg_return_true:  }
        { \group_insert_after:N \prg_return_false: }
    \group_end:
  }
\prg_new_protected_conditional:Npnn \@@_check_prevsec:nn #1#2 { F }
  {
    \group_begin:
      \bool_set_true:N \l_@@_integer_bool
      \zrefcheck_get_asint:nnn {#1} { zc@abssec  } { \l_@@_lbl_int }
      \zrefcheck_get_asint:nnn {#2} { zc@abssec  } { \l_@@_ref_int }
      \zrefcheck_get_asint:nnn {#1} { zc@abschap } { \l_@@_lbl_b_int }
      \zrefcheck_get_asint:nnn {#2} { zc@abschap } { \l_@@_ref_b_int }
      \bool_lazy_and:nnTF
        { \l_@@_integer_bool }
        {
          \int_compare_p:nNn
            { \l_@@_lbl_b_int } = { \l_@@_ref_b_int } &&
          \int_compare_p:nNn
            { \l_@@_lbl_int } = { \l_@@_ref_int - 1 } &&
          ! \int_compare_p:nNn { \l_@@_lbl_int } = { 0 } &&
          ! \int_compare_p:nNn { \l_@@_ref_int } = { 0 }
        }
        { \group_insert_after:N \prg_return_true:  }
        { \group_insert_after:N \prg_return_false: }
    \group_end:
  }
\prg_new_protected_conditional:Npnn \@@_check_secsafter:nn #1#2 { F }
  {
    \group_begin:
      \bool_set_true:N \l_@@_integer_bool
      \zrefcheck_get_asint:nnn {#1} { zc@abssec  } { \l_@@_lbl_int }
      \zrefcheck_get_asint:nnn {#2} { zc@abssec  } { \l_@@_ref_int }
      \zrefcheck_get_asint:nnn {#1} { zc@abschap } { \l_@@_lbl_b_int }
      \zrefcheck_get_asint:nnn {#2} { zc@abschap } { \l_@@_ref_b_int }
      \bool_lazy_and:nnTF
        { \l_@@_integer_bool }
        {
          \int_compare_p:nNn
            { \l_@@_lbl_b_int } = { \l_@@_ref_b_int } &&
          \int_compare_p:nNn
            { \l_@@_lbl_int } > { \l_@@_ref_int } &&
          ! \int_compare_p:nNn { \l_@@_lbl_int } = { 0 }
        }
        { \group_insert_after:N \prg_return_true:  }
        { \group_insert_after:N \prg_return_false: }
    \group_end:
  }
\prg_new_protected_conditional:Npnn \@@_check_secsbefore:nn #1#2 { F }
  {
    \group_begin:
      \bool_set_true:N \l_@@_integer_bool
      \zrefcheck_get_asint:nnn {#1} { zc@abssec  } { \l_@@_lbl_int }
      \zrefcheck_get_asint:nnn {#2} { zc@abssec  } { \l_@@_ref_int }
      \zrefcheck_get_asint:nnn {#1} { zc@abschap } { \l_@@_lbl_b_int }
      \zrefcheck_get_asint:nnn {#2} { zc@abschap } { \l_@@_ref_b_int }
      \bool_lazy_and:nnTF
        { \l_@@_integer_bool }
        {
          \int_compare_p:nNn
            { \l_@@_lbl_b_int } = { \l_@@_ref_b_int } &&
          \int_compare_p:nNn
            { \l_@@_lbl_int } < { \l_@@_ref_int } &&
          ! \int_compare_p:nNn { \l_@@_lbl_int } = { 0 } &&
          ! \int_compare_p:nNn { \l_@@_ref_int } = { 0 }
        }
        { \group_insert_after:N \prg_return_true:  }
        { \group_insert_after:N \prg_return_false: }
    \group_end:
  }
%    \end{macrocode}
% \end{macro}
%
%
%    \begin{macrocode}
%</package>
%    \end{macrocode}
%
% \PrintIndex
%
% \end{implementation}
%
