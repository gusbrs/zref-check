% \iffalse meta-comment
%
% File: zref-check.dtx
% Copyright (C) 2021 Gustavo Barros
%
% It may be distributed and/or modified under the conditions of the
% LaTeX Project Public License (LPPL), either version 1.3c of this
% license or (at your option) any later version.  The latest version
% of this license is in the file:
%
%    https://www.latex-project.org/lppl.txt
%
% The released version of this work is available from CTAN.
%
% -----------------------------------------------------------------------
%
% The development version of the work can be found at
%
%    https://github.com/gusbrs/zref-check
%
% for those people who are interested.
%
% -----------------------------------------------------------------------
%
% \fi
%
% \iffalse
%<*driver>
\documentclass{l3doc}
% Have \GetFileInfo pick up date and version data
\usepackage{zref-check}
\begin{document}
  \DocInput{zref-check.dtx}
\end{document}
%</driver>
% \fi
%
%
% \GetFileInfo{zref-check.sty}
%
% \title{^^A
%   Package \pkg{zref-check}^^A
%   \thanks{This file describes \fileversion,
%     last revised \filedate.}^^A
% }
%
% \author{^^A
%  Gustavo Barros^^A
%  \thanks{^^A
%    \url{https://github.com/gusbrs/zref-check}^^A
%   }^^A
% }
%
% \date{Released \filedate}
%
% \maketitle
%
%
% \begin{documentation}
%
% \end{documentation}
%
%
% \begin{implementation}
%
% \part{\pkg{zref-check} implementation}
%
% Start the \pkg{DocStrip} guards.
%    \begin{macrocode}
%<*package>
%    \end{macrocode}
%
% Identify the internal prefix (\LaTeX3 \pkg{DocStrip} convention).
%    \begin{macrocode}
%<@@=zrefcheck>
%    \end{macrocode}
%
% \section{Initial setup}
%
%    \begin{macrocode}
\ProvidesExplPackage {zref-check} {2021-07-21} {0.1.0}
  {Flexible cross-references with contextual checks based on zref}
%    \end{macrocode}
%
% \section{Dependencies}
%
%    \begin{macrocode}
\RequirePackage { zref-user }
\RequirePackage { zref-abspage }
\RequirePackage { ifdraft }
%    \end{macrocode}
%
%
% \section{\pkg{zref} setup}
%
% \begin{variable}{\g_@@_abschap_int, \g_@@_abssec_int}
%   Provide absolute counters for section and chapter, and respective zref
%   properties, so that we can make checks about relation of chapters/sections
%   regardless of internal counters, since we don't get those for the
%   unnumbered (starred) ones.  About the proper place to make the hooks for
%   this purpose, see \url{https://tex.stackexchange.com/q/605533/105447},
%   thanks Ulrike.
%    \begin{macrocode}
\int_new:N \g_@@_abschap_int
\int_new:N \g_@@_abssec_int
%    \end{macrocode}
% \end{variable}
%
%
% If the documentclass does not define \cs{chapter} the only thing that
% happens is that the chapter counter is never incremented, and the section
% one never reset.
%    \begin{macrocode}
\AddToHook { cmd / chapter / before }
  {
    \int_gincr:N \g_@@_abschap_int
    \int_zero:N \g_@@_abssec_int
  }
\zref@newprop { abschap } [0] { \int_use:N \g_@@_abschap_int }
\zref@addprop \ZREF@mainlist { abschap }
%    \end{macrocode}
%
%    \begin{macrocode}
\AddToHook { cmd / section / before }
  { \int_gincr:N \g_@@_abssec_int }
\zref@newprop { abssec } [0] { \int_use:N \g_@@_abssec_int }
\zref@addprop \ZREF@mainlist { abssec }
%    \end{macrocode}
%
%
% This is the list of properties to be used by \pkg{zref-check}, that is, the
% list of properties the references and targets store.  This is the minimum
% set required, more properties may be added according to options.
%    \begin{macrocode}
\zref@newlist { zrefcheck }
\zref@addprops { zrefcheck }
  {
    abspage ,
    abschap ,
    abssec ,
    page
  }
%    \end{macrocode}
%
%
% \section{Plumbing}
%
% \subsection{Messages}
%
% \begin{macro}{\@@_message:nnnn, \@@_message:nnnx}
%    \begin{macrocode}
\cs_new_protected:Npn \@@_message:nnnn #1#2#3#4
  {
    \use:c { msg_ \l_@@_msglevel_tl :nnnnn }
      { zref-check } {#1} {#2} {#3} {#4}
  }
\cs_generate_variant:Nn \@@_message:nnnn { nnnx }
%    \end{macrocode}
% \end{macro}
%
%    \begin{macrocode}
\msg_new:nnn { zref-check } { check-failed }
  { Failed~check~'#1'~for~label~'#2' \\ on~page~#3~on~input~line~\msg_line_number:. }
\msg_new:nnn { zref-check } { double-check }
  { Double-check~'#1'~for~label~'#2' \\ on~page~#3~on~input~line~\msg_line_number:. }
%    \end{macrocode}
%
%    \begin{macrocode}
\msg_new:nnn { zref-check } { check-missing }
  { Check~'#1'~not~defined~on~input~line~\msg_line_number:. }
\msg_new:nnn { zref-check } { property-undefined }
  { Property~'#1'~not~defined~on~input~line~\msg_line_number:. }
\msg_new:nnn { zref-check } { property-not-in-label }
  { Label~'#1'~has~no~property~'#2'~on~input~line~\msg_line_number:. }
\msg_new:nnn { zref-check } { property-not-integer }
  { Property~'#1'~for~label~'#2'~not~an~integer~on~input~line~\msg_line_number:. }
%    \end{macrocode}
%
%    \begin{macrocode}
\msg_new:nnn { zref-check } { hyperref-preamble-only }
  {
    Option~'hyperref'~only~available~in~the~preamble. \\
    Use~the~starred~version~of~'\protect\zrcheck'~instead.
  }
\msg_new:nnn { zref-check } { missing-hyperref }
  { Missing~'hyperref'~package. \\ Setting~'hyperref=false'. }
\msg_new:nnn { zref-check } { ignore-document-only }
  {
    Option~'ignore'~only~available~in~the~document. \\
    Use~option~'msglevel'~instead.
  }
%    \end{macrocode}
%
%
% \subsection{Options}
%
% \verb|hyperref| option
%
% \begin{variable}{\l_@@_use_hyperref_bool, \l_@@_warn_hyperref_bool}
%    \begin{macrocode}
\bool_new:N \l_@@_use_hyperref_bool
\bool_new:N \l_@@_warn_hyperref_bool
\keys_define:nn { zref-check }
  {
    hyperref .choice: ,
    hyperref / auto .code:n =
      {
        \bool_set_true:N \l_@@_use_hyperref_bool
        \bool_set_false:N \l_@@_warn_hyperref_bool
      } ,
    hyperref / true .code:n =
      {
        \bool_set_true:N \l_@@_use_hyperref_bool
        \bool_set_true:N \l_@@_warn_hyperref_bool
      } ,
    hyperref / false .code:n =
      {
        \bool_set_false:N \l_@@_use_hyperref_bool
        \bool_set_false:N \l_@@_warn_hyperref_bool
      } ,
    hyperref .default:n = auto
  }
%    \end{macrocode}
% \end{variable}
%
%    \begin{macrocode}
\AtBeginDocument
  {
    \@ifpackageloaded { hyperref }
      {
        \bool_if:NT \l_@@_use_hyperref_bool
          {
            \RequirePackage { zref-hyperref }
            \zref@addprop { zrefcheck } { anchor }
          }
      }
      {
        \bool_if:NT \l_@@_warn_hyperref_bool
          { \msg_warning:nn { zref-check } { missing-hyperref } }
        \bool_set_false:N \l_@@_use_hyperref_bool
      }
    \keys_define:nn { zref-check }
      {
        hyperref .code:n =
          { \msg_warning:nn { zref-check } { hyperref-preamble-only } }
      }
  }
%    \end{macrocode}
%
% \verb|msglevel| option
% \begin{variable}{\l_@@_msglevel_tl}
%    \begin{macrocode}
\tl_new:N \l_@@_msglevel_tl
\keys_define:nn { zref-check }
  {
    msglevel .choice: ,
    msglevel / warn .code:n =
      { \tl_set:Nn \l_@@_msglevel_tl { warning } } ,
    msglevel / info .code:n =
      { \tl_set:Nn \l_@@_msglevel_tl { info } } ,
    msglevel / none .code:n =
      { \tl_set:Nn \l_@@_msglevel_tl { none } } ,
    msglevel / obeydraft .code:n =
      {
        \ifdraft
          { \tl_set:Nn \l_@@_msglevel_tl { info } }
          { \tl_set:Nn \l_@@_msglevel_tl { warning } }
      } ,
    msglevel / obeyfinal .code:n =
      {
        \ifoptionfinal
          { \tl_set:Nn \l_@@_msglevel_tl { warning } }
          { \tl_set:Nn \l_@@_msglevel_tl { info } }
      } ,
%    \end{macrocode}
% \verb|ignore| alias for \verb|msglevel=none|
%    \begin{macrocode}
    ignore .code:n =
      { \msg_warning:nn { zref-check } { ignore-document-only } }
  }
%    \end{macrocode}
% \end{variable}
%
%    \begin{macrocode}
\AtBeginDocument
  {
    \keys_define:nn { zref-check }
      {
        ignore .code:n =
          { \keys_set:nn { zref-check } { msglevel = none } }
      }
  }
%    \end{macrocode}
%
%
% \verb|onpage| option
% \begin{variable}{\l_@@_msgonpage_bool}
%    \begin{macrocode}
\bool_new:N \l_@@_msgonpage_bool
\keys_define:nn { zref-check }
  {
    onpage .choice: ,
    onpage / labelseq .code:n =
      {
        \bool_set_false:N \l_@@_msgonpage_bool
      } ,
    onpage / msg .code:n =
      {
        \bool_set_true:N \l_@@_msgonpage_bool
      } ,
    onpage / obeydraft .code:n =
      {
        \ifdraft
          { \bool_set_false:N \l_@@_msgonpage_bool }
          { \bool_set_true:N \l_@@_msgonpage_bool }
      } ,
    onpage / obeyfinal .code:n =
      {
        \ifoptionfinal
          { \bool_set_true:N \l_@@_msgonpage_bool }
          { \bool_set_false:N \l_@@_msgonpage_bool }
      }
  }
%    \end{macrocode}
% \end{variable}
%
%
% \verb|closerange| option
% \begin{variable}{\l_@@_close_range_int}
%    \begin{macrocode}
\int_new:N \l_@@_close_range_int
\keys_define:nn { zref-check }
  {
    closerange .int_set:N = \l_@@_close_range_int ,
  }
%    \end{macrocode}
% \end{variable}
%
%
% Set load-time default values
%    \begin{macrocode}
\keys_set:nn { zref-check }
  {
    hyperref   = auto ,
    msglevel   = warn ,
    onpage     = labelseq ,
    closerange = 5
  }
%    \end{macrocode}
%
% Process load-time package options
% https://tex.stackexchange.com/a/15840
%    \begin{macrocode}
\RequirePackage { l3keys2e }
\ProcessKeysOptions { zref-check }
%    \end{macrocode}
%
%
% \begin{macro}{\zrchecksetup}
%   Provide \cs{zrchecksetup}.
%    \begin{macrocode}
\NewDocumentCommand \zrchecksetup { m }
  { \keys_set:nn { zref-check } {#1} }
%    \end{macrocode}
% \end{macro}
%
%
% \subsection{Position on page}
%
% Method for determining relative position within the page: the sequence in
% which the labels get shipped out, inferred from the sequence in which the
% labels occur in the \file{.aux} file.
%
% Some relevant info about the sequence of things:
% \url{https://tex.stackexchange.com/a/120978} and \verb|texdoc lthooks|,
% section ``Hooks provided by \verb|\begin{document}|''.
%
%
% One first attempt at this was to use \cs{zref@newlabel}, which is the macro
% in which \pkg{zref} stores the label information in the aux file.  When the
% \file{.aux} file is read at the beginning of the compilation, this macro is
% expanded for each of the labels.  So, by redefining this macro we can feed a
% variable (a L3 sequence), and then do what it usually does, which is to
% define each label with the internal macro \cs{@newl@bel}, when the
% \file{.aux} file is read.
%
% Patching this macro for this is not possible.  First, \cs{zref@newlabel} is
% one of those ``commands that look ahead'' mentioned in \verb|ltcmdhooks|
% documentation.  Indeed, \cs{@newl@bel} receives 3 arguments, and
% \cs{zref@newlabel} just passes the first, the following two will be scanned
% ahead.  Second, the \verb|ltcmdhooks| hooks are not actually available when
% the \file{.aux} file is read, they come only after \verb|\begin{document}|.
% Hence, redefinition would be the only alternative.  My attempts at this
% ended up registered at \url{https://tex.stackexchange.com/a/604744}.  But
% the best result in these lines was:
%
% \begin{verbatim}
% \ZREF@Robust\edef\zref@newlabel#1{
%   \noexpand\seq_gput_right:Nn \noexpand\g__zrefcheck_auxfile_lblseq_seq {#1}
%   \noexpand\@newl@bel{\ZREF@RefPrefix}{#1}
% }
% \end{verbatim}
%
%
% However, better than the above is to just read it from the \file{.aux} file
% directly, no need to redefine anything.  That's what David Carlisle's answer
% at \url{https://tex.stackexchange.com/a/147705} does.  This answer has
% actually been converted into the package \pkg{listlbls} by Norbert Melzer,
% but it is made to work with regular labels, not with \pkg{zref}'s.  And it
% also does not really expose the information in a retrievable way (as far as
% I can tell).  So, the below is adapted from Carlisle's answer's technique (a
% poor man's version of it...).
%
% There is some subtlety here as to whether this approach makes it safe for us
% to read the labels at this point without \cs{zref@wrapper@babel}.  The
% common wisdom is that babel's shorthands are only active after
% \verb|\begin{document}| (e.g., \url{https://tex.stackexchange.com/a/98897}).
% Alas, it is more complicated than that.  Babel's documentation says (in
% section 9.5 Shorthands): ``To prevent problems with the loading of other
% packages after babel we reset the catcode of the character to the original
% one at the end of the package and of each language file (except with
% KeepShorthandsActive).  It is re-activate[d] again at
% \verb|\begin{document}|.  We also need to make sure that the shorthands are
% active during the processing of the \file{.aux} file.  Otherwise some
% citations may give unexpected results in the printout when a shorthand was
% used in the optional argument of \verb|\bibitem| for example.''  This is
% done with \verb|\if@filesw \immediate\write\@mainaux{...}|.  In other words,
% the catcode change is written in the \file{.aux} file itself!  Indeed, if
% you inspect the file, you'll find them there.  Besides, there is still the
% ominous ``except with KeepShorthandsActive''.
%
% However, the \emph{method} I'm using here is not quite the same as the usual
% run of the \file{.aux} file, because I'm actively discarding the lines for
% which the first token is not equal to \cs{zref@newlabel}.  I have tested the
% famous sensitive case for this: \pkg{babel} with \verb|french| and labels
% with colons.  And I was able to retrieve the information as expected.  Well,
% \emph{if} \verb|KeepShorthandsActive| is enabled \emph{with \verb|french|}
% and we load the package \emph{after babel} things do break, but not quite
% because of the colons in the labels.  Even \pkg{siunitx} breaks in the same
% conditions...
%
% For reference: About what are valid characters for use in labels:
% \url{https://tex.stackexchange.com/a/18312}.  About some problems with
% active colons: \url{https://tex.stackexchange.com/a/89470}.
%
% About the difference between L3 strings and token lists, see
% \url{https://tex.stackexchange.com/a/446381}, in particular Joseph Wright's
% comment: ``Strings are for data that will never be typeset, for example file
% names, identifiers, etc.: if the material may be used in typesetting, it
% should be a token list.''  See also moewe's (CW) answer in the same lines.
% Which suggests using L3 strings for the reference labels might be a good
% catch all approach, and possibly more robust.  David Carlisle's comment
% about \pkg{inputenc} is a caveat (see
% \url{https://tex.stackexchange.com/q/446123#comment1516961_446381}).
% Still... let's stick to tradition as long as it works, \pkg{zref} already
% does a great job here anyway.
%
%
% \begin{variable}{\g_@@_auxfile_lblseq_prop}
%    \begin{macrocode}
\prop_new:N \g_@@_auxfile_lblseq_prop
%    \end{macrocode}
% \end{variable}
%
%    \begin{macrocode}
\tl_set:Nn \g_tmpa_tl { \c_sys_jobname_str .aux }
\file_if_exist:nT { \g_tmpa_tl }
  {
%    \end{macrocode}
% Retrieve the information from the \file{.aux} file, and store it in a
% property list, so that the sequence can be retrieved in key-value fashion.
%    \begin{macrocode}
    \ior_open:Nn \g_tmpa_ior { \g_tmpa_tl }
    \group_begin:
      \int_zero:N \l_tmpa_int
      \tl_clear:N \l_tmpa_tl
      \tl_clear:N \l_tmpb_tl
      \bool_set_false:N \l_tmpa_bool
      \ior_map_variable:NNn \g_tmpa_ior \l_tmpa_tl
        {
          \tl_map_variable:NNn \l_tmpa_tl \l_tmpb_tl
            {
              \tl_if_eq:NnTF \l_tmpb_tl { \zref@newlabel }
                {
%    \end{macrocode}
% Found a \cs{zref@label}, signal it.
%    \begin{macrocode}
                  \bool_set_true:N \l_tmpa_bool
                }
                {
                  \bool_if:NTF \l_tmpa_bool
                    {
                      \bool_set_false:N \l_tmpa_bool
                      \int_incr:N \l_tmpa_int
                      \prop_gput:Nxx \g_@@_auxfile_lblseq_prop
                        { \l_tmpb_tl } { \int_use:N \l_tmpa_int }
                    }
                    {
%    \end{macrocode}
% If there is not a match of the first token with \cs{zref@newlabel}, break
% the loop and discard the rest of the line, to ensure no babel calls to
% \cs{catcode} in the \file{.aux} file get expanded.  This also breaks the
% loop and discards the rest of the \cs{zref@newlabel} lines after we got the
% label we wanted, since we reset \cs{l_tmpa_bool} in the \verb|T| branch.
%    \begin{macrocode}
                      \tl_map_break:
                    }
                }
            }
        }
    \group_end:
    \ior_close:N \g_tmpa_ior
  }
%    \end{macrocode}
%
%
%
% The alternate method I had considered (more than that...) for this was using
% yx coordinates supplied by \pkg{zref}'s \verb|savepos| module.  However,
% this approach brought in a number of complexities, including the need to
% patch either \cs{zref@label} or \cs{ZREF@label}.  In addition, the technique
% was at the bottom fundamentally flawed.  Ulrike Fischer was very much right
% when she said that ``structure and position are two different beasts''
% (\url{https://github.com/ho-tex/zref/issues/12#issuecomment-880022576}).  It
% is true that the checks based on it behaved decently, in normal
% circumstances, and except for outrageous label placement by the user, it
% would return the expected results.  We don't really need exact coordinates
% to decide ``above/below''.  Besides, it would do an exact job for the
% dedicated target macros of this package.  However, I could not conceive a
% situation where the \verb|yx| criterion would perform clearly better than
% the \verb|labeseq| one.  And, if that's the case, and considering the
% complications it brings, this check was a slippery slope.  All in all, I've
% decided to drop it.
%
%
%
% \subsection{Counter}
%
% We need a dedicated counter for the labels generated by the checks and
% targets.  The value of the counter is not relevant, we just need it to be
% able to set proper anchors with \cs{refstepcounter}.  And, since I couldn't
% find a \cs{refstepcounter} equivalent in L3, we use a standard 2e counter
% here.  I'm also using the technique to ensure the counter is never reset
% that is used by \file{zref-abspage.sty} and \cs{zref@require@unique}.  I
% don't know why it is needed, but if Oberdiek does it, there must be a
% reason.  In any case, the requirements are the same, we need numbers ensured
% to be \emph{unique} in the counter.
%
%    \begin{macrocode}
\begingroup
  \let \@addtoreset \ltx@gobbletwo
  \newcounter { zrefcheck }
\endgroup
\setcounter { zrefcheck } { 0 }
%    \end{macrocode}
%
%
%
% \subsection{Label formats}
%
% \begin{macro}{\@@_check_lblfmt:n}
%   \marg{<check id int>}
%    \begin{macrocode}
\cs_new:Npn \@@_check_lblfmt:n #1
  { zrefcheck@ \int_use:N #1 }
%    \end{macrocode}
% \end{macro}
%
% \begin{macro}{\@@_end_lblfmt:n}
%   \marg{<label>}
%    \begin{macrocode}
\cs_new:Npn \@@_end_lblfmt:n #1
  { #1 @zrefcheck }
%    \end{macrocode}
% \end{macro}
%
%
%
% \subsection{Property values}
%
%
% \begin{macro}{\zrefcheck_get_astl:nnn}
%   A convenience function to retrieve property values from labels.  Uses
%   \cs{g_@@_auxfile_lblseq_prop} for \verb|lblseq|, and calls
%   \cs{zref@extractdefault} for everything else.
%
%   We cannot use the ``return value'' of \cs{@@_get_astl:nnn} or
%   \cs{@@_get_asint:nnn} directly, because we need to use the retrieved
%   property values as arguments in the checks, however we use here a number
%   of non-expandable operations.  Hence, we receive a local \verb|tl/int|
%   variable as third argument and set that, so that it is available (and
%   expandable) at the place of use.  For this reason, we do not group here,
%   because we are passing a local variable around, but it is expected this
%   function will be called within a group.
%
%   I'm returning \cs{c_empty_tl} in case of failure to find the intended
%   property value (explicitly in \cs{zref@extractdefault}, but that is also
%   what \cs{tl_clear:N} does).
%
%   \begin{syntax}
%     \cs{zrefcheck_get_astl:nnn} \Arg{label} \Arg{prop} \Arg{tl var}
%   \end{syntax}
%    \begin{macrocode}
\cs_new:Npn \zrefcheck_get_astl:nnn #1#2#3
  {
    \tl_clear:N #3
    \tl_if_eq:nnTF {#2} { lblseq }
      {
        \prop_get:NnNF \g_@@_auxfile_lblseq_prop {#1} #3
          {
            \msg_warning:nnnn { zref-check }
              { property-not-in-label } {#1} {#2}
          }
      }
      {
%    \end{macrocode}
% There are three things we need to check to ensure the information we are
% trying to retrieve here exists: the existence of \Arg{label}, the existence
% of \Arg{prop}, and whether the particular label being queried actually
% contains the property.  If that's all in place, the value is passed to the
% checks, and it's their responsibility to verify the consistency of this
% value.
%
% The existence of the label is an user facing issue, and a warning for this
% is placed in \cs{@@_zrcheck:nnnnn} (and done with \cs{zref@refused}).  We do
% check here though for definition with \cs{zref@ifrefundefined} and silently
% do nothing if it is undefined, to reduce irrelevant warnings in a fresh
% compilation round.  The other two are more ``internal'' problems, either
% some problem with the checks, or with the configuration of \pkg{zref} for
% their consumption.
%    \begin{macrocode}
        \zref@ifrefundefined {#1}
          {}
          {
            \zref@ifpropundefined {#2}
              { \msg_warning:nnnn { zref-check } { property-undefined } {#2} }
              {
                \zref@ifrefcontainsprop {#1} {#2}
                  {
                    \tl_set:Nx #3
                      { \zref@extractdefault {#1} {#2} { \c_empty_tl } }
                  }
                  {
                    \msg_warning:nnnn
                      { zref-check } { property-not-in-label } {#1} {#2}
                  }
              }
          }
      }
  }
%    \end{macrocode}
% \end{macro}
%
%
% \begin{variable}{\l_@@_integer_bool}
%   \cs{zrefcheck_get_asint:nnn} is a very convenient wrapper around the more
%   general \cs{zrefcheck_get_astl:nnn}, since almost always we'll be wanting
%   to compare numbers in the checks.  However, it is quite hard for it to
%   ensure an integer is \emph{always} returned in the case of errors.  And
%   those do occur, even in a well structured document (e.g., in a first round
%   of compilation).  To complicate things, the L3 integer predicates are
%   \emph{very} sensitive to receiving any other kind of data, and they
%   \emph{scream}.  To handle this \cs{zrefcheck_get_asint:nnn} uses
%   \cs{l_@@_integer_bool} to signal if an integer could not be returned.  To
%   use this function always set \cs{l_@@_integer_bool} to true first, then
%   call it as much as you need.  If any of these calls got is returning
%   anything which is not an integer, \cs{l_@@_integer_bool} will have been
%   set to false, and you should check that this hasn't happened before
%   actually comparing the integers (\cs{bool_lazy_and:nnTF} is your friend).
%    \begin{macrocode}
\bool_new:N \l_@@_integer_bool
%    \end{macrocode}
% \end{variable}
%
% \begin{variable}{\l_@@_propval_tl}
%    \begin{macrocode}
\tl_new:N \l_@@_propval_tl
%    \end{macrocode}
% \end{variable}
%
% \begin{macro}{\zrefcheck_get_asint:nnn}
%   \begin{syntax}
%     \cs{zrefcheck_get_asint:nnn} \Arg{label} \Arg{prop} \Arg{int var}
%   \end{syntax}
%    \begin{macrocode}
\cs_new:Npn \zrefcheck_get_asint:nnn #1#2#3
  {
    \zrefcheck_get_astl:nnn {#1} {#2} { \l_@@_propval_tl }
    \@@_is_integer:nTF { \l_@@_propval_tl }
      {
%    \end{macrocode}
% Make it an integer data type.
%    \begin{macrocode}
        \int_set:Nn #3 { \int_eval:n { \l_@@_propval_tl } }
      }
      {
        \bool_set_false:N \l_@@_integer_bool
        \zref@ifrefundefined {#1}
%    \end{macrocode}
% Keep silent if ref is undefined to reduce irrelevant warnings in a fresh
% compilation round.  Again, this is also not the point to check for undefined
% references, that's a task for \cs{@@_zrcheck:nnnnn}.
%    \begin{macrocode}
          { }
          {
            \msg_warning:nnnn { zref-check }
              { property-not-integer } {#2} {#1}
          }
      }
  }
%    \end{macrocode}
% \end{macro}
%
% \begin{macro}{\@@_is_integer:n}
%    \begin{macrocode}
\prg_new_conditional:Npnn \@@_is_integer:n #1 { p, T, F, TF }
  {
    \tl_if_empty:oTF {#1}
%    \end{macrocode}
% Empty tl is also not an integer.
%    \begin{macrocode}
      { \prg_return_false: }
      {
%    \end{macrocode}
% Thanks egreg: \url{https://tex.stackexchange.com/a/244405}.
%    \begin{macrocode}
        \tl_if_blank:oTF { \__int_to_roman:w -0#1 }
          { \prg_return_true:  }
          { \prg_return_false: }
      }
  }
%    \end{macrocode}
% \end{macro}
%
%
%
% \section{\cs{zrcheck}}
%
% \begin{macro}{\zrcheck}
%   The \marg{text} argument of \cs{zrcheck} should not be long, since
%   \cs{hyperlink} cannot receive a long argument.  Besides, there is no
%   reason for it to be.  Note, also, that hyperlinks crossing page boundaries
%   have some known issues: \url{https://tex.stackexchange.com/a/182769},
%   \url{https://tex.stackexchange.com/a/54607},
%   \url{https://tex.stackexchange.com/a/179907}.
%
%   \begin{syntax}
%     \cs{zrcheck}\meta{*}\oarg{options}\marg{labels}\oarg{checks}\marg{text}
%   \end{syntax}
%
%    \begin{macrocode}
\NewDocumentCommand \zrcheck
  { s O { } > { \SplitList { , } } m > { \SplitList { , } } O { } m }
  { \zref@wrapper@babel \@@_zrcheck:nnnnn {#3} {#1} {#2} {#4} {#5} }
%    \end{macrocode}
% \end{macro}
%
%
% \begin{variable}
%   {
%     \g_@@_id_int ,
%     \l_@@_checkbeg_tl ,
%     \l_@@_checkend_tl ,
%     \l_@@_link_label_tl ,
%     \l_@@_link_anchor_tl ,
%     \l_@@_link_star_tl
%   }
%    \begin{macrocode}
\int_new:N \g_@@_id_int
\tl_new:N \l_@@_checkbeg_tl
\tl_new:N \l_@@_checkend_tl
\tl_new:N \l_@@_link_label_tl
\tl_new:N \l_@@_link_anchor_tl
\bool_new:N \l_@@_link_star_tl
%    \end{macrocode}
% \end{variable}
%
% \begin{macro}{\@@_zrcheck:nnnnn}
%   An intermediate internal function, which places \Arg{labels} as first
%   argument, so that it can be protected by \cs{zref@wrapper@babel}.  This is
%   more or less what the definition of \cs{zref} in \file{zref-user.sty} does
%   for this.
%
%   \begin{syntax}
%     \cs{@@_zrcheck:nnnnn} \Arg{labels} \Arg{*} \Arg{options} \Arg{checks} \Arg{text}
%   \end{syntax}
%
%    \begin{macrocode}
\cs_new_protected:Npn \@@_zrcheck:nnnnn #1#2#3#4#5
  {
    \group_begin:
%    \end{macrocode}
% Process local options.
%    \begin{macrocode}
      \keys_set:nn { zref-check } {#3}
%    \end{macrocode}
% Names of the labels for this zrefcheck call.
%    \begin{macrocode}
      \int_gincr:N \g_@@_id_int
      \tl_set:Nx \l_@@_checkbeg_tl
        { \@@_check_lblfmt:n { \g_@@_id_int } }
      \tl_set:Nx \l_@@_checkend_tl
        { \@@_end_lblfmt:n { \l_@@_checkbeg_tl } }
%    \end{macrocode}
% Set checkbeg label.
%    \begin{macrocode}
      \refstepcounter { zrefcheck }
      \zref@labelbylist { \l_@@_checkbeg_tl } { zrefcheck }
%    \end{macrocode}
% Typeset \marg{text}, with hyperlink when appropriate.
%
% Even though the first argument can receive a list of labels, there is no
% meaningful way to set links to multiple targets.  Hence, only the first one
% is considered for hyperlinking.
%    \begin{macrocode}
      \tl_set:Nn \l_@@_link_label_tl { \tl_item:nn {#1} {1} }
      \bool_set:Nn \l_@@_link_star_tl {#2}
      \zref@ifrefundefined { \l_@@_link_label_tl }
%    \end{macrocode}
% If the reference is undefined, just typeset.
%    \begin{macrocode}
        {#5}
        {
          \bool_if:nTF
            {
              \l_@@_use_hyperref_bool &&
              ! \l_@@_link_star_tl
            }
            {
              \exp_args:Nx \zrefcheck_get_astl:nnn
                { \l_@@_link_label_tl }
                { anchor } { \l_@@_link_anchor_tl }
              \hyperlink { \l_@@_link_anchor_tl } {#5}
            }
            {#5}
        }
%    \end{macrocode}
% Set checkend label.
%    \begin{macrocode}
      \refstepcounter { zrefcheck }
      \zref@labelbylist { \l_@@_checkend_tl } { zrefcheck }
%    \end{macrocode}
% Check definition.
%    \begin{macrocode}
      \tl_map_function:nN {#1} \zref@refused
%    \end{macrocode}
% Run the checks.
%    \begin{macrocode}
      \@@_run_checks:nnV {#4} {#1} { \l_@@_checkbeg_tl }
    \group_end:
  }
%    \end{macrocode}
% \end{macro}
%
%
%
% \section{Targets}
%
% \begin{macro}{\zrctarget}
%   \begin{syntax}
%     \cs{zrctarget}\marg{label}\marg{text}
%   \end{syntax}
%    \begin{macrocode}
\NewDocumentCommand \zrctarget { m +m }
  {
    \refstepcounter { zrefcheck }
    \zref@wrapper@babel \zref@labelbylist {#1} { zrefcheck }
    #2
    \refstepcounter { zrefcheck }
    \zref@wrapper@babel
      \zref@labelbylist { \@@_end_lblfmt:n {#1} } { zrefcheck }
  }
%    \end{macrocode}
% \end{macro}
%
% \begin{environment}{zrcregion}
%   \begin{syntax}
%     |\begin{zrcregion}|\marg{label}
%       |...|
%     |\end{zrcregion}|
%   \end{syntax}
%    \begin{macrocode}
\NewDocumentEnvironment {zrcregion} { m }
  {
    \refstepcounter { zrefcheck }
    \zref@wrapper@babel \zref@labelbylist {#1} { zrefcheck }
  }
  {
    \refstepcounter { zrefcheck }
    \zref@wrapper@babel
      \zref@labelbylist { \@@_end_lblfmt:n {#1} } { zrefcheck }
  }
%    \end{macrocode}
% \end{environment}
%
%
%
% \section{Checks}
%
% What is needed for a check to work?
%
% First, a conditional/predicate function defined with:
%
% \cs{prg_new_conditional:Npnn} \cs{@@_check_\meta{check}:nn}
%   \verb|#1#2| \verb|{ F }|
%
% \noindent where \meta{check}, it the name of the check, the first argument
% is the \Arg{label} and the second the \Arg{reference}.  The existence of the
% check is verified by the existence of the function with this name-scheme
% (and signatures).  Of course, this function must return either
% \cs{prg_return_true:} or \cs{prg_return_false:}.  Of course, you can define
% other variants if you need internally, and may do a protected definition, if
% it is needed for the content of the check, just what the package does expect
% and verifies is existence of the \verb|:nnF| variant.
%
% Note that the naming convention of the checks adopts the perspective of the
% \meta{reference}.  That is, the ``before'' check should return true if the
% \meta{label} occurs before the ``reference''.
%
% The checks it does are expected to retrieve \pkg{zref}'s label information
% with \cs{zrefcheck_get_astl:nnn} or \cs{zrefcheck_get_asint:nnn}.  Also,
% technically speaking, the \meta{reference} argument is also a label,
% actually a pair of them, as set by \cs{zrcheck}.  For the ``labels'', any
% \pkg{zref} property in \pkg{zref}'s main list is available, the
% ``references'' store the properties in the \verb|zrefcheck| list.  Besides
% those, there is also the \verb|lblseq| (fake) property (for either
% ``labels'' or ``references''), stored in \cs{g_@@_auxfile_lblseq_prop}.
%
% Second, the required properties of labels and references must be duly
% registered for \pkg{zref}.  This can be done with \cs{zref@newprop},
% \cs{zref@addprop} and friends, as usual.
%
%
% \subsection{Running}
%
% \begin{macro}{\@@_run_checks:nnn, \@@_run_checks:nnV}
%   \begin{syntax}
%     \cs{@@_run_checks:nnn}
%       \Arg{list of checks} \Arg{list of labels} \Arg{reference}
%   \end{syntax}
%    \begin{macrocode}
\cs_new:Npn \@@_run_checks:nnn #1#2#3
  {
    \group_begin:
      \tl_map_inline:nn {#2}
        {
          \tl_map_inline:nn {#1}
            { \@@_do_check:nnn {####1} {##1} {#3} }
        }
    \group_end:
  }
\cs_generate_variant:Nn \@@_run_checks:nnn { nnV }
%    \end{macrocode}
% \end{macro}
%
%
% \begin{variable}
%   {
%     \l_@@_passedcheck_bool ,
%     \l_@@_onpage_bool ,
%     \c_@@_onpage_checks_seq
%   }
%    \begin{macrocode}
\bool_new:N \l_@@_passedcheck_bool
\bool_new:N \l_@@_onpage_bool
\seq_new:N \c_@@_onpage_checks_seq
\seq_set_from_clist:Nn \c_@@_onpage_checks_seq
  { above , below , before , after }
%    \end{macrocode}
% \end{variable}
%
%
% Variant not provided by \pkg{expl3}.
%    \begin{macrocode}
\cs_generate_variant:Nn \exp_args:Nnno { Nnoo }
%    \end{macrocode}
%
% \begin{macro}{\@@_do_check:nnn}
%   \begin{syntax}
%     \cs{@@_do_check:nnn} \Arg{check} \Arg{label beg} \Arg{reference beg}
%   \end{syntax}
%    \begin{macrocode}
\cs_new:Npn \@@_do_check:nnn #1#2#3
  {
    \group_begin:
%    \end{macrocode}
% \meta{label beg} may be defined or not, it is arbitrary user input.  Whether
% this is the case is checked in \cs{@@_zrcheck:nnnnn}, and due warning
% already ensues.  And there is no point in checking ``relative position'' of
% an undefined label.  Hence, in the absence of \verb|#2|, we do nothing at
% all here.
%    \begin{macrocode}
      \zref@ifrefundefined {#2}
        {}
        {
          \bool_set_true:N \l_@@_passedcheck_bool
          \bool_set_false:N \l_@@_onpage_bool
          \cs_if_exist:cTF { @@_check_ #1 :nnF }
            {
%    \end{macrocode}
% ``label beg'' vs ``reference beg''.
%    \begin{macrocode}
              \use:c { @@_check_ #1 :nnF }
                {#2} {#3}
                { \bool_set_false:N \l_@@_passedcheck_bool }
%    \end{macrocode}
% ``label beg'' vs ``reference end''.
%    \begin{macrocode}
              \exp_args:Nnno \use:c { @@_check_ #1 :nnF }
                {#2} { \@@_end_lblfmt:n {#3} }
                { \bool_set_false:N \l_@@_passedcheck_bool }
%    \end{macrocode}
% ``label end'' \emph{may} have been created by the target commands.
%    \begin{macrocode}
              \zref@ifrefundefined { \@@_end_lblfmt:n {#2} }
                {}
                {
%    \end{macrocode}
% ``label end'' vs ``reference beg''.
%    \begin{macrocode}
                  \exp_args:Nno \use:c { @@_check_ #1 :nnF }
                    { \@@_end_lblfmt:n {#2} } {#3}
                    { \bool_set_false:N \l_@@_passedcheck_bool }
%    \end{macrocode}
% ``label end'' vs ``reference end''.
%    \begin{macrocode}
                  \exp_args:Nnoo \use:c { @@_check_ #1 :nnF }
                    { \@@_end_lblfmt:n {#2} }
                    { \@@_end_lblfmt:n {#3} }
                    { \bool_set_false:N \l_@@_passedcheck_bool }
                }
%    \end{macrocode}
% Handle option \verb|onpage=msg|.  This is only granted for tests which
% perform ``within this page'' checks (\verb|above|, \verb|below|,
% \verb|before|, \verb|after|) \emph{and} if any of the two by two checks uses
% a ``within this page'' comparison.  If both conditions are met, signal.
%    \begin{macrocode}
              \seq_if_in:NnT \c_@@_onpage_checks_seq {#1}
                {
                  \@@_check_thispage:nnT
                    {#2} {#3}
                    { \bool_set_true:N \l_@@_onpage_bool }
                  \@@_check_thispage:nnT
                    {#2} { \@@_end_lblfmt:n {#3} }
                    { \bool_set_true:N \l_@@_onpage_bool }
                  \zref@ifrefundefined { \@@_end_lblfmt:n {#2} }
                    {}
                    {
                      \@@_check_thispage:nnT
                        { \@@_end_lblfmt:n {#2} } {#3}
                        { \bool_set_true:N \l_@@_onpage_bool }
                      \@@_check_thispage:nnT
                        { \@@_end_lblfmt:n {#2} }
                        { \@@_end_lblfmt:n {#3} }
                        { \bool_set_true:N \l_@@_onpage_bool }
                    }
                }
              \bool_if:NTF \l_@@_passedcheck_bool
                {
                  \bool_if:nT
                    {
                      \l_@@_msgonpage_bool &&
                      \l_@@_onpage_bool
                    }
                    {
                      \@@_message:nnnx { double-check } {#1} {#2}
                        { \zref@extractdefault {#3} {page} {'unknown'} }
                    }
                }
                {
                  \@@_message:nnnx { check-failed } {#1} {#2}
                    { \zref@extractdefault {#3} {page} {'unknown'} }
                }
            }
            { \msg_warning:nnn { zref-check } { check-missing } {#1} }
        }
    \group_end:
  }
%    \end{macrocode}
% \end{macro}
%
%
% \subsection{Definitions}
%
% \begin{variable}{\l_@@_lbl_int, \l_@@_ref_int}
%   More readable scratch variables for the tests.
%    \begin{macrocode}
\int_new:N \l_@@_lbl_int
\int_new:N \l_@@_ref_int
%    \end{macrocode}
% \end{variable}
%
%
% \subsubsection{This page}
%
% \begin{macro}{\@@_check_thispage:nn}
%    \begin{macrocode}
\prg_new_protected_conditional:Npnn \@@_check_thispage:nn #1#2 { T, F , TF }
  {
    \group_begin:
      \bool_set_true:N \l_@@_integer_bool
      \zrefcheck_get_asint:nnn {#1} { abspage } { \l_@@_lbl_int }
      \zrefcheck_get_asint:nnn {#2} { abspage } { \l_@@_ref_int }
      \bool_lazy_and:nnTF
        { \l_@@_integer_bool }
        {
          \int_compare_p:nNn
            { \l_@@_lbl_int } = { \l_@@_ref_int } &&
%    \end{macrocode}
% `0' is the default value of \verb|abspage| property, and this value should
% not happen normally for this property, since even the first page, after it
% gets shipped out, will store `1'.  So, if we do find `0' here, better signal
% something is wrong.
%    \begin{macrocode}
            ! \int_compare_p:nNn { \l_@@_lbl_int } = { 0 } &&
            ! \int_compare_p:nNn { \l_@@_ref_int } = { 0 }
        }
        { \group_insert_after:N \prg_return_true:  }
        { \group_insert_after:N \prg_return_false: }
    \group_end:
  }
%    \end{macrocode}
% \end{macro}
%
%
% \subsubsection{On page}
%
% \begin{macro}{\@@_check_above:nn, \@@_check_below:nn}
%    \begin{macrocode}
\prg_new_protected_conditional:Npnn \@@_check_above:nn #1#2 { F , TF }
  {
    \group_begin:
      \@@_check_thispage:nnTF {#1} {#2}
        {
          \bool_set_true:N \l_@@_integer_bool
          \zrefcheck_get_asint:nnn {#1} { lblseq } { \l_@@_lbl_int }
          \zrefcheck_get_asint:nnn {#2} { lblseq } { \l_@@_ref_int }
          \bool_lazy_and:nnTF
            { \l_@@_integer_bool }
            {
              \int_compare_p:nNn
                { \l_@@_lbl_int } < { \l_@@_ref_int } &&
              ! \int_compare_p:nNn { \l_@@_lbl_int } = { 0 } &&
              ! \int_compare_p:nNn { \l_@@_ref_int } = { 0 }
            }
            { \group_insert_after:N \prg_return_true:  }
            { \group_insert_after:N \prg_return_false: }
        }
        { \group_insert_after:N \prg_return_false: }
    \group_end:
  }
\prg_new_protected_conditional:Npnn \@@_check_below:nn #1#2 { F , TF }
  {
    \@@_check_thispage:nnTF {#1} {#2}
      {
        \@@_check_above:nnTF {#1} {#2}
          { \prg_return_false: }
          { \prg_return_true:  }
      }
      { \prg_return_false: }
  }
%    \end{macrocode}
% \end{macro}
%
%
% \subsubsection{Before / After}
%
% \begin{macro}{\@@_check_before:nn, \@@_check_after:nn}
%    \begin{macrocode}
\prg_new_protected_conditional:Npnn \@@_check_before:nn #1#2 { F }
  {
    \@@_check_pagesbefore:nnTF {#1} {#2}
      { \prg_return_true: }
      {
        \@@_check_above:nnTF {#1} {#2}
          { \prg_return_true:  }
          { \prg_return_false: }
      }
  }
\prg_new_protected_conditional:Npnn \@@_check_after:nn #1#2 { F }
  {
    \@@_check_pagesafter:nnTF {#1} {#2}
      { \prg_return_true: }
      {
        \@@_check_below:nnTF {#1} {#2}
          { \prg_return_true:  }
          { \prg_return_false: }
      }
  }
%    \end{macrocode}
% \end{macro}
%
%
% \subsubsection{Pages}
%
% \begin{macro}
%   {
%     \@@_check_nextpage:nn ,
%     \@@_check_prevpage:nn ,
%     \@@_check_pagesbefore:nn ,
%     \@@_check_ppbefore:nn ,
%     \@@_check_pagesafter:nn ,
%     \@@_check_ppafter:nn ,
%     \@@_check_facing:nn
%   }
%    \begin{macrocode}
\prg_new_protected_conditional:Npnn \@@_check_nextpage:nn #1#2 { F }
  {
    \group_begin:
      \bool_set_true:N \l_@@_integer_bool
      \zrefcheck_get_asint:nnn {#1} { abspage } { \l_@@_lbl_int }
      \zrefcheck_get_asint:nnn {#2} { abspage } { \l_@@_ref_int }
      \bool_lazy_and:nnTF
        { \l_@@_integer_bool }
        {
          \int_compare_p:nNn
            { \l_@@_lbl_int } = { \l_@@_ref_int + 1 } &&
%    \end{macrocode}
% Ditto.
%    \begin{macrocode}
          ! \int_compare_p:nNn { \l_@@_lbl_int } = { 0 } &&
          ! \int_compare_p:nNn { \l_@@_ref_int } = { 0 }
        }
        { \group_insert_after:N \prg_return_true:  }
        { \group_insert_after:N \prg_return_false: }
    \group_end:
  }
\prg_new_protected_conditional:Npnn \@@_check_prevpage:nn #1#2 { F }
  {
    \group_begin:
      \bool_set_true:N \l_@@_integer_bool
      \zrefcheck_get_asint:nnn {#1} { abspage } { \l_@@_lbl_int }
      \zrefcheck_get_asint:nnn {#2} { abspage } { \l_@@_ref_int }
      \bool_lazy_and:nnTF
        { \l_@@_integer_bool }
        {
          \int_compare_p:nNn
            { \l_@@_lbl_int } = { \l_@@_ref_int - 1 } &&
%    \end{macrocode}
% Ditto.
%    \begin{macrocode}
          ! \int_compare_p:nNn { \l_@@_lbl_int } = { 0 } &&
          ! \int_compare_p:nNn { \l_@@_ref_int } = { 0 }
        }
        { \group_insert_after:N \prg_return_true:  }
        { \group_insert_after:N \prg_return_false: }
    \group_end:
  }
\prg_new_protected_conditional:Npnn \@@_check_pagesbefore:nn #1#2 { F , TF }
  {
    \group_begin:
      \bool_set_true:N \l_@@_integer_bool
      \zrefcheck_get_asint:nnn {#1} { abspage } { \l_@@_lbl_int }
      \zrefcheck_get_asint:nnn {#2} { abspage } { \l_@@_ref_int }
      \bool_lazy_and:nnTF
        { \l_@@_integer_bool }
        {
          \int_compare_p:nNn
            { \l_@@_lbl_int } < { \l_@@_ref_int } &&
%    \end{macrocode}
% Ditto.
%    \begin{macrocode}
          ! \int_compare_p:nNn { \l_@@_lbl_int } = { 0 } &&
          ! \int_compare_p:nNn { \l_@@_ref_int } = { 0 }
        }
        { \group_insert_after:N \prg_return_true:  }
        { \group_insert_after:N \prg_return_false: }
    \group_end:
  }
\cs_new_eq:NN \@@_check_ppbefore:nnF \@@_check_pagesbefore:nnF
\prg_new_protected_conditional:Npnn \@@_check_pagesafter:nn #1#2 { F , TF }
  {
    \group_begin:
      \bool_set_true:N \l_@@_integer_bool
      \zrefcheck_get_asint:nnn {#1} { abspage } { \l_@@_lbl_int }
      \zrefcheck_get_asint:nnn {#2} { abspage } { \l_@@_ref_int }
      \bool_lazy_and:nnTF
        { \l_@@_integer_bool }
        {
          \int_compare_p:nNn
            { \l_@@_lbl_int } > { \l_@@_ref_int } &&
%    \end{macrocode}
% Ditto.
%    \begin{macrocode}
          ! \int_compare_p:nNn { \l_@@_lbl_int } = { 0 } &&
          ! \int_compare_p:nNn { \l_@@_ref_int } = { 0 }
        }
        { \group_insert_after:N \prg_return_true:  }
        { \group_insert_after:N \prg_return_false: }
    \group_end:
  }
\cs_new_eq:NN \@@_check_ppafter:nnF \@@_check_pagesafter:nnF
\prg_new_protected_conditional:Npnn \@@_check_facing:nn #1#2 { F }
  {
    \group_begin:
      \bool_set_true:N \l_@@_integer_bool
      \zrefcheck_get_asint:nnn {#1} { abspage } { \l_@@_lbl_int }
      \zrefcheck_get_asint:nnn {#2} { abspage } { \l_@@_ref_int }
      \bool_lazy_and:nnTF
        { \l_@@_integer_bool }
        {
%    \end{macrocode}
% There exists no ``facing'' page if the document is not twoside.
%    \begin{macrocode}
          \legacy_if_p:n { @twoside } &&
%    \end{macrocode}
% Now we test ``facing''.
%    \begin{macrocode}
          (
            (
              \int_if_odd_p:n { \l_@@_ref_int } &&
              \int_compare_p:nNn
                { \l_@@_lbl_int } = { \l_@@_ref_int - 1 }
            ) ||
            (
              \int_if_even_p:n { \l_@@_ref_int } &&
              \int_compare_p:nNn
                { \l_@@_lbl_int } = { \l_@@_ref_int + 1 }
            )
          ) &&
%    \end{macrocode}
% Ditto.
%    \begin{macrocode}
          ! \int_compare_p:nNn { \l_@@_lbl_int } = { 0 } &&
          ! \int_compare_p:nNn { \l_@@_ref_int } = { 0 }
        }
        { \group_insert_after:N \prg_return_true:  }
        { \group_insert_after:N \prg_return_false: }
    \group_end:
  }
%    \end{macrocode}
% \end{macro}
%
%
% \subsubsection{Close / Far}
%
% \begin{macro}{\@@_check_close:nn, \@@_check_far:nn}
%    \begin{macrocode}
\prg_new_protected_conditional:Npnn \@@_check_close:nn #1#2 { F , TF }
  {
    \group_begin:
      \bool_set_true:N \l_@@_integer_bool
      \zrefcheck_get_asint:nnn {#1} { abspage } { \l_@@_lbl_int }
      \zrefcheck_get_asint:nnn {#2} { abspage } { \l_@@_ref_int }
      \bool_lazy_and:nnTF
        { \l_@@_integer_bool }
        {
          \int_compare_p:nNn
            { \int_abs:n { \l_@@_lbl_int - \l_@@_ref_int } }
            <
            { \l_@@_close_range_int + 1 } &&
%    \end{macrocode}
% Ditto.
%    \begin{macrocode}
          ! \int_compare_p:nNn { \l_@@_lbl_int } = { 0 } &&
          ! \int_compare_p:nNn { \l_@@_ref_int } = { 0 }
        }
        { \group_insert_after:N \prg_return_true:  }
        { \group_insert_after:N \prg_return_false: }
    \group_end:
  }
\prg_new_protected_conditional:Npnn \@@_check_far:nn #1#2 { F }
  {
    \@@_check_close:nnTF {#1} {#2}
      { \prg_return_false: }
      { \prg_return_true:  }
  }
%    \end{macrocode}
% \end{macro}
%
%
% \subsubsection{Chapter}
%
% \begin{macro}
%   {
%     \@@_check_thischap:nn ,
%     \@@_check_nextchap:nn ,
%     \@@_check_prevchap:nn ,
%     \@@_check_chapsafter:nn ,
%     \@@_check_chapsbefore:nn
%   }
%    \begin{macrocode}
\prg_new_protected_conditional:Npnn \@@_check_thischap:nn #1#2 { F }
  {
    \group_begin:
      \bool_set_true:N \l_@@_integer_bool
      \zrefcheck_get_asint:nnn {#1} { abschap } { \l_@@_lbl_int }
      \zrefcheck_get_asint:nnn {#2} { abschap } { \l_@@_ref_int }
      \bool_lazy_and:nnTF
        { \l_@@_integer_bool }
        {
          \int_compare_p:nNn
            { \l_@@_lbl_int } = { \l_@@_ref_int } &&
%    \end{macrocode}
% `0' is the default value of \verb|abschap| property, and means here no
% \cs{chapter} has yet been issued, therefore it cannot be ``this chapter'',
% nor ``the next chapter'', nor ``the previous chapter'', it is just ``no
% chapter''.  Note, however, that a statement about a ``future'' chapter does
% not require the ``current'' one to exist.
%    \begin{macrocode}
          ! \int_compare_p:nNn { \l_@@_lbl_int } = { 0 } &&
          ! \int_compare_p:nNn { \l_@@_ref_int } = { 0 }
        }
        { \group_insert_after:N \prg_return_true:  }
        { \group_insert_after:N \prg_return_false: }
    \group_end:
  }
\prg_new_protected_conditional:Npnn \@@_check_nextchap:nn #1#2 { F }
  {
    \group_begin:
      \bool_set_true:N \l_@@_integer_bool
      \zrefcheck_get_asint:nnn {#1} { abschap } { \l_@@_lbl_int }
      \zrefcheck_get_asint:nnn {#2} { abschap } { \l_@@_ref_int }
      \bool_lazy_and:nnTF
        { \l_@@_integer_bool }
        {
          \int_compare_p:nNn
            { \l_@@_lbl_int } = { \l_@@_ref_int + 1 } &&
%    \end{macrocode}
% Ditto (though redundant, in this case).
%    \begin{macrocode}
          ! \int_compare_p:nNn { \l_@@_lbl_int } = { 0 }
        }
        { \group_insert_after:N \prg_return_true:  }
        { \group_insert_after:N \prg_return_false: }
    \group_end:
  }
\prg_new_protected_conditional:Npnn \@@_check_prevchap:nn #1#2 { F }
  {
    \group_begin:
      \bool_set_true:N \l_@@_integer_bool
      \zrefcheck_get_asint:nnn {#1} { abschap } { \l_@@_lbl_int }
      \zrefcheck_get_asint:nnn {#2} { abschap } { \l_@@_ref_int }
      \bool_lazy_and:nnTF
        { \l_@@_integer_bool }
        {
          \int_compare_p:nNn
            { \l_@@_lbl_int } = { \l_@@_ref_int - 1 } &&
%    \end{macrocode}
% Ditto.
%    \begin{macrocode}
          ! \int_compare_p:nNn { \l_@@_lbl_int } = { 0 } &&
          ! \int_compare_p:nNn { \l_@@_ref_int } = { 0 }
        }
        { \group_insert_after:N \prg_return_true:  }
        { \group_insert_after:N \prg_return_false: }
    \group_end:
  }
\prg_new_protected_conditional:Npnn \@@_check_chapsafter:nn #1#2 { F }
  {
    \group_begin:
      \bool_set_true:N \l_@@_integer_bool
      \zrefcheck_get_asint:nnn {#1} { abschap } { \l_@@_lbl_int }
      \zrefcheck_get_asint:nnn {#2} { abschap } { \l_@@_ref_int }
      \bool_lazy_and:nnTF
        { \l_@@_integer_bool }
        {
          \int_compare_p:nNn
            { \l_@@_lbl_int } > { \l_@@_ref_int } &&
%    \end{macrocode}
% Ditto (though redundant, in this case).
%    \begin{macrocode}
          ! \int_compare_p:nNn { \l_@@_lbl_int } = { 0 }
        }
        { \group_insert_after:N \prg_return_true:  }
        { \group_insert_after:N \prg_return_false: }
    \group_end:
  }
\prg_new_protected_conditional:Npnn \@@_check_chapsbefore:nn #1#2 { F }
  {
    \group_begin:
      \bool_set_true:N \l_@@_integer_bool
      \zrefcheck_get_asint:nnn {#1} { abschap } { \l_@@_lbl_int }
      \zrefcheck_get_asint:nnn {#2} { abschap } { \l_@@_ref_int }
      \bool_lazy_and:nnTF
        { \l_@@_integer_bool }
        {
          \int_compare_p:nNn
            { \l_@@_lbl_int } < { \l_@@_ref_int } &&
%    \end{macrocode}
% Ditto.
%    \begin{macrocode}
          ! \int_compare_p:nNn { \l_@@_lbl_int } = { 0 } &&
          ! \int_compare_p:nNn { \l_@@_ref_int } = { 0 }
        }
        { \group_insert_after:N \prg_return_true:  }
        { \group_insert_after:N \prg_return_false: }
    \group_end:
  }
%    \end{macrocode}
% \end{macro}
%
%
% \subsubsection{Section}
%
% \begin{macro}
%   {
%     \@@_check_thissec:nn ,
%     \@@_check_nextsec:nn ,
%     \@@_check_prevsec:nn ,
%     \@@_check_secsafter:nn ,
%     \@@_check_secsbefore:nn
%   }
%    \begin{macrocode}
\prg_new_protected_conditional:Npnn \@@_check_thissec:nn #1#2 { F }
  {
    \group_begin:
      \bool_set_true:N \l_@@_integer_bool
      \zrefcheck_get_asint:nnn {#1} { abssec } { \l_@@_lbl_int }
      \zrefcheck_get_asint:nnn {#2} { abssec } { \l_@@_ref_int }
      \bool_lazy_and:nnTF
        { \l_@@_integer_bool }
        {
          \int_compare_p:nNn
            { \l_@@_lbl_int } = { \l_@@_ref_int } &&
%    \end{macrocode}
% `0' is the default value of \verb|abssec| property, and means here no
% \cs{section} has yet been issued since its counter has been reset, which
% occurs at the beginning of the document and at every chapter.  Hence, as is
% the case for chapters, `0' is just ``not a section''.  The same observation
% about the need of the ``current'' section to exist to be able to refer to a
% ``future'' one also holds.
%    \begin{macrocode}
          ! \int_compare_p:nNn { \l_@@_lbl_int } = { 0 } &&
          ! \int_compare_p:nNn { \l_@@_ref_int } = { 0 }
        }
        { \group_insert_after:N \prg_return_true:  }
        { \group_insert_after:N \prg_return_false: }
    \group_end:
  }
\prg_new_protected_conditional:Npnn \@@_check_nextsec:nn #1#2 { F }
  {
    \group_begin:
      \bool_set_true:N \l_@@_integer_bool
      \zrefcheck_get_asint:nnn {#1} { abssec } { \l_@@_lbl_int }
      \zrefcheck_get_asint:nnn {#2} { abssec } { \l_@@_ref_int }
      \bool_lazy_and:nnTF
        { \l_@@_integer_bool }
        {
          \int_compare_p:nNn
            { \l_@@_lbl_int } = { \l_@@_ref_int + 1 } &&
%    \end{macrocode}
% Ditto (though redundant, in this case).
%    \begin{macrocode}
          ! \int_compare_p:nNn { \l_@@_lbl_int } = { 0 }
        }
        { \group_insert_after:N \prg_return_true:  }
        { \group_insert_after:N \prg_return_false: }
    \group_end:
  }
\prg_new_protected_conditional:Npnn \@@_check_prevsec:nn #1#2 { F }
  {
    \group_begin:
      \bool_set_true:N \l_@@_integer_bool
      \zrefcheck_get_asint:nnn {#1} { abssec } { \l_@@_lbl_int }
      \zrefcheck_get_asint:nnn {#2} { abssec } { \l_@@_ref_int }
      \bool_lazy_and:nnTF
        { \l_@@_integer_bool }
        {
          \int_compare_p:nNn
            { \l_@@_lbl_int } = { \l_@@_ref_int - 1 } &&
%    \end{macrocode}
% Ditto.
%    \begin{macrocode}
          ! \int_compare_p:nNn { \l_@@_lbl_int } = { 0 } &&
          ! \int_compare_p:nNn { \l_@@_ref_int } = { 0 }
        }
        { \group_insert_after:N \prg_return_true:  }
        { \group_insert_after:N \prg_return_false: }
    \group_end:
  }
\prg_new_protected_conditional:Npnn \@@_check_secsafter:nn #1#2 { F }
  {
    \group_begin:
      \bool_set_true:N \l_@@_integer_bool
      \zrefcheck_get_asint:nnn {#1} { abssec } { \l_@@_lbl_int }
      \zrefcheck_get_asint:nnn {#2} { abssec } { \l_@@_ref_int }
      \bool_lazy_and:nnTF
        { \l_@@_integer_bool }
        {
          \int_compare_p:nNn
            { \l_@@_lbl_int } > { \l_@@_ref_int } &&
%    \end{macrocode}
% Ditto (though redundant, in this case).
%    \begin{macrocode}
          ! \int_compare_p:nNn { \l_@@_lbl_int } = { 0 }
        }
        { \group_insert_after:N \prg_return_true:  }
        { \group_insert_after:N \prg_return_false: }
    \group_end:
  }
\prg_new_protected_conditional:Npnn \@@_check_secsbefore:nn #1#2 { F }
  {
    \group_begin:
      \bool_set_true:N \l_@@_integer_bool
      \zrefcheck_get_asint:nnn {#1} { abssec } { \l_@@_lbl_int }
      \zrefcheck_get_asint:nnn {#2} { abssec } { \l_@@_ref_int }
      \bool_lazy_and:nnTF
        { \l_@@_integer_bool }
        {
          \int_compare_p:nNn
            { \l_@@_lbl_int } < { \l_@@_ref_int } &&
%    \end{macrocode}
% Ditto.
%    \begin{macrocode}
          ! \int_compare_p:nNn { \l_@@_lbl_int } = { 0 } &&
          ! \int_compare_p:nNn { \l_@@_ref_int } = { 0 }
        }
        { \group_insert_after:N \prg_return_true:  }
        { \group_insert_after:N \prg_return_false: }
    \group_end:
  }
%    \end{macrocode}
% \end{macro}
%
%
%    \begin{macrocode}
%</package>
%    \end{macrocode}
%
% \end{implementation}
