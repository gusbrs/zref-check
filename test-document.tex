\documentclass{book}

\usepackage[user,counter,hyperref,titleref]{zref}
\usepackage{zref-check}

\usepackage{graphicx}
\usepackage{econlipsum}
\usepackage{hyperref}
\usepackage[all]{hypcap}


% \usepackage[french]{babel}

\begin{document}

\zlabel{firstthing}

\frontmatter

\tableofcontents{}

\chapter*{Preface}
\zlabel{cha:preface}

Hello World!


\mainmatter

\chapter*{Introduction}
\zlabel{cha:introduction}

\chapter{Chapter 1}
\zlabel{cha:chapter-1}

\econ[1]

\clearpage{}

\begin{figure}
  \centering
  \includegraphics[width=.6\textwidth]{example-image-a}
  \caption{Figure 1}
  \zlabel{fig:figure-1}
\end{figure}


Teste1\phantomsection\zlabel{test1}. \zrefcheck{test2}[below]{mesma linha}.
Teste2\phantomsection\zlabel{test2}.

\clearpage{}


% É assim que constrói um hyperlink pra uma zlabel! Essa macro aparece na
% documentação do hyperref, mas sem explicação alguma. O primeiro argumento é
% a ‘anchor’, propriedade que o módulo zref-hyperref disponibiliza.  :-)
\hyperlink{figure.1.1}{\zref{fig:figure-1}}


\zrefcheck{fig:figure-1}[
  facing,
  % teste
  prevpage
  ]{alguma coisa e tal}

\clearpage{}

\begin{figure}
  \centering
  \includegraphics[width=.6\textwidth]{example-image-c}
  \caption{Figure 3}
  \zlabel{fig:figure-3}
\end{figure}

\chapter{Chapter 2}
\zlabel{cha:chapter-2}

\econ[2]

\section{section-2-1}
\zlabel{sec:section-2-1}

\clearpage{}
\zrefcheck{ fig:figure-2 }[ above ]{alguma coisa e tal}


\begin{figure}[hb]
  \centering
  \caption{Figure 2}
  \includegraphics[width=.6\textwidth]{example-image-b}
  \zlabel{fig:figure-2}
\end{figure}


\hyperlink{figure.2.1}{\zref{fig:figure-2}}

\chapter{Chapter 3}
\zlabel{cha:chapter-3}

\econ[3]





\chapter{Chapter 4}
\zlabel{cha:chapter-4}

\econ[4]




\backmatter



\chapter{Bibliography}
\zlabel{cha:bibliography}



\end{document}